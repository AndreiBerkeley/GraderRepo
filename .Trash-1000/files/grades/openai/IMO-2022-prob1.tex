% Auto-graded by OpenAI
I would rate this solution a 3.

The first mistake in the student's solution is in the analysis of the possible values of \( m \) for a given \( k \). The student correctly identifies that if \( k \leq n \), then the longest chain containing the \( k \)th coin from the left is of length \( k \), but incorrectly states that if \( n < k \leq 2n \), then the longest chain containing the \( k \)th coin from the left is of length \( 2n - k + 1 \). This is incorrect because the longest chain containing the \( k \)th coin from the left is actually of length \( 2n - k \) (not \( 2n - k + 1 \)). 

Therefore, the correct expression for \( m \) when \( n < k \leq 2n \) should be \( m = 2n - k \) instead of \( m = 2n - k + 1 \). This mistake leads to an incorrect conclusion about the pairs \( (n, k) \) for which the leftmost \( n \) coins will all be of the same type.
