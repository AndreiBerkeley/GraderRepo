Let's first consider the case when $n=1$. In this case, there is only one square on the board, so $k(C)=1$ for any configuration $C$.

Now, let's consider the case when $n=3$. In this case, the board looks like the following:
\[
\begin{array}{|c|c|c|}
\hline
& & \\
\hline
& & \\
\hline
& & \\
\hline
\end{array}
\]
For $n=3$, there is only one possible configuration of dominoes that covers all but one square. It looks like this:
\[
\begin{array}{|c|c|c|}
\hline
& & \\
\hline
& & \\
\hline
& & \\
\hline
\end{array}
\]
In this case, we can slide the domino in two different ways to uncover the remaining square. So, $k(C)=2$ for $n=3$.

Now, let's consider the case when $n=5$. In this case, the board looks like the following:
\[
\begin{array}{|c|c|c|c|c|}
\hline
& & & & \\
\hline
& & & & \\
\hline
& & & & \\
\hline
& & & & \\
\hline
& & & & \\
\hline
\end{array}
\]
For $n=5$, there are two possible configurations of dominoes that cover all but one square. They look like this:
\[
\begin{array}{|c|c|c|c|c|}
\hline
& & & & \\
\hline
& & & & \\
\hline
& & & & \\
\hline
& & & & \\
\hline
& & & & \\
\hline
\end{array}
\]
and
\[
\begin{array}{|c|c|c|c|c|}
\hline
& & & & \\
\hline
& & & & \\
\hline
& & & & \\
\hline
& & & & \\
\hline
& & & & \\
\hline
\end{array}
\]
In the first configuration, we can slide the domino in four different ways to uncover the remaining square. In the second configuration, we can slide the domino in two different ways to uncover the remaining square. So, $k(C)=4+2=6$ for $n=5$.

From the above analysis, we can see that $k(C)$ depends on $n$ as follows:
\[k(C) = 
\begin{cases} 
1 & \text{if } n=1 \\
2 & \text{if } n=3 \\
2n & \text{if } n>3
\end{cases}
\]

Therefore, the possible values of $k(C)$ as a function of $n$ are $1$, $2$, and $2n$ for $n>3$.
