Let  

\[
\mathcal S_{k,m}(z)=\Bigl\{\,
(x_{1},\dots ,x_{k},y_{1},\dots ,y_{m})\in \Bbb Z ^{k+m}\; ;\;
1\le x_{1}\le\cdots \le x_{k}\le z,\;
1\le y_{1}\le\cdots \le y_{m}\le z\Bigr\}.
\]

For a fixed positive integer \(z\) the number of such pairs of
non-decreasing \(k\)- and \(m\)-tuples equals  

\[
|\mathcal S_{k,m}(z)|=\binom{z+k-1}{k}\binom{z+m-1}{m},
\tag{1}
\]

because \(\binom{z+k-1}{k}\) counts the multisets of cardinality \(k\)
chosen from \(\{1,\dots ,z\}\) and the second factor does the same for
cardinality \(m\).
With this notation

\[
f(n)=\sum_{z=1}^{n}\binom{z+k-1}{k}\binom{z+m-1}{m}.
\tag{2}
\]

The task is to show that \(f(n)\) is a polynomial in \(n\) all of whose
coefficients are non–negative.

--------------------------------------------------------------------
1.  A convenient polynomial basis  
--------------------------------------------------------------------
For non–negative integers \(r\) put  

\[
\llbracket n \\ r \rrbracket=\binom{n}{r}
       =\frac{n(n-1)\dots(n-r+1)}{r!}.
\]

The family \(\bigl\{\llbracket n \\ r \rrbracket\bigr\}_{r\ge 0}\) (the
“binomial‐coefficient basis’’) is a basis of the space of all
polynomials in \(n\):
every polynomial

\[
P(n)\quad(\deg P\le d)
\]

can be written uniquely in the
Newton expansion  

\[
P(n)=\sum_{r=0}^d
        \bigl(\Delta^{\,r}P\bigr)(0)\;
        \llbracket n \\ r\rrbracket ,
\tag{3}
\]

where \(\Delta\) is the forward–difference operator
\((\Delta P)(n)=P(n+1)-P(n)\).
The coefficients in this expansion are  

\[
c_r:=\bigl(\Delta^{\,r}P\bigr)(0)
     =\sum_{j=0}^{r}(-1)^{r-j}\binom{r}{j}P(j).
\tag{4}
\]

--------------------------------------------------------------------
2.  Expanding the summand in the Newton basis  
--------------------------------------------------------------------
Fix \(k,m\ge 1\) and set  

\[
g(z):=\binom{z+k-1}{k}\binom{z+m-1}{m}\qquad(z\ge 0).
\]

Because \(g\) is a polynomial in \(z\) of degree \(k+m\),
(3) gives

\[
g(z)=\sum_{r=0}^{k+m}a_{r}\llbracket z\\ r\rrbracket ,
\qquad a_{r}=\bigl(\Delta^{\,r}g\bigr)(0).
\tag{5}
\]

We next show that every \(a_{r}\) is a non–negative integer.

--------------------------------------------------------------------
3.  The coefficients \(a_{r}\) are non–negative  
--------------------------------------------------------------------
Observe first that \(g(0)=0\).  Moreover \(g(1)=1\) only when
\(k=m=1\); in every case \(g(z)\) is strictly increasing in \(z\) because
allowing a larger maximal value offers (weakly) more possibilities for
all coordinates.  Consequently  

\[
\Delta g(z)=g(z+1)-g(z)\ge 0
\qquad\text{for every }z\ge 0 .
\tag{6}
\]

Apply (6) repeatedly:
all higher forward differences of \(g\) are non–negative,

\[
\bigl(\Delta^{\,r}g\bigr)(z)\;\ge\;0
\qquad\text{for every }r\ge 1,\;z\ge 0.
\tag{7}
\]

(In words: the sequence \(g(0),g(1),g(2),\dots\) is {\it convex} in the
sense of finite differences.)  In particular  

\[
a_{r}=\bigl(\Delta^{\,r}g\bigr)(0)\ge 0
\qquad\bigl(0\le r\le k+m\bigr),
\tag{8}
\]

so every coefficient appearing in (5) is a non–negative integer.

--------------------------------------------------------------------
4.  Summing over \(z\)  
--------------------------------------------------------------------
Insert (5) into (2):

\[
f(n)=\sum_{z=1}^{n}g(z)
    =\sum_{z=1}^{n}\,
      \sum_{r=0}^{k+m}a_{r}\llbracket z\\ r\rrbracket
    =\sum_{r=0}^{k+m}a_{r}
      \sum_{z=1}^{n}\llbracket z\\ r\rrbracket .
\tag{9}
\]

The inner sum can be evaluated explicitly:

\[
\sum_{z=1}^{n}\llbracket z\\ r\rrbracket
 =\sum_{z=0}^{n}\llbracket z\\ r\rrbracket
          -\llbracket 0\\ r\rrbracket
 =\binom{n+1}{r+1}-0
 =\llbracket n+1\\ r+1\rrbracket ,
\tag{10}
\]

so (9) becomes  

\[
f(n)=\sum_{r=0}^{k+m}a_{r}\;
      \llbracket n+1\\ r+1\rrbracket .
\tag{11}
\]

--------------------------------------------------------------------
5.  Positivity of the final coefficients  
--------------------------------------------------------------------
Equation (11) shows that \(f(n)\) is a linear combination of the
polynomials
\(\llbracket n+1\\ r+1\rrbracket\;(0\le r\le k+m)\).
Each polynomial \(\llbracket n+1\\ r+1\rrbracket
      =\tfrac1{(r+1)!}(n+1)n(n-1)\dots(n-r+2)\)
has, when expanded, only non-negative coefficients in the usual power
basis \(\{1,n,n^{2},\dots\}\).
Together with \(a_{r}\ge 0\) from (8) this proves that the right–hand
side of (11) is itself a polynomial in \(n\) whose every coefficient is
non-negative.

--------------------------------------------------------------------
6.  Conclusion  
--------------------------------------------------------------------
For every pair of positive integers \(k,m\) the function  

\[
f(n)=\sum_{z=1}^{n}\binom{z+k-1}{k}\binom{z+m-1}{m}
\qquad(n=1,2,\dots)
\]

admits the representation  

\[
\boxed{\;
f(n)=\displaystyle\sum_{r=0}^{k+m}
        a_{r}\binom{n+1}{r+1}},
\]

where \(a_{r}=\bigl(\Delta^{\,r}g\bigr)(0)\) are non–negative integers.
Because every binomial coefficient \(\binom{n+1}{r+1}\) expands into a
polynomial in \(n\) with non-negative coefficients, \(f(n)\) itself is a
polynomial in \(n\) with all coefficients non-negative, as was to be
shown.