Let \( P(x,y) \) be the assertion that \( f(xf(x) + y) = f(y) + x^2 \).

From \( P(0,0) \), we have
\[ f(0) = f(0) \]
which is a tautology.

From \( P(0,y) \), we have
\[ f(y) = f(y) \]
which is also a tautology.

From \( P(x,0) \), we have
\[ f(xf(x)) = f(0) + x^2 \]
\[ f(xf(x)) = f(0) + x^2 \]
\[ f(xf(x)) = f(0) + x^2 \]
\[ f(xf(x)) = f(0) + x^2 \]
which implies that \( f \) is surjective.

Let \( a \) be a rational number such that \( f(a) = 0 \). Then, from \( P(a,0) \), we have
\[ f(0) = f(0) + a^2 \]
which implies that \( a = 0 \).

From \( P(x,-xf(x)) \), we have
\[ f(0) = f(-xf(x)) + x^2 \]
\[ f(-xf(x)) = -x^2 \]
which implies that \( f \) is injective.

From \( P(1,0) \), we have
\[ f(f(1)) = f(0) + 1 \]
\[ f(f(1)) = 1 \]
which implies that \( f(1) = 1 \).

From \( P(1,y) \), we have
\[ f(f(1) + y) = f(y) + 1 \]
\[ f(y + 1) = f(y) + 1 \]
which implies that \( f(x) = x \) for all rational numbers \( x \).

Therefore, the only function that satisfies the given functional equation is \( f(x) = x \) for all rational numbers \( x \).
