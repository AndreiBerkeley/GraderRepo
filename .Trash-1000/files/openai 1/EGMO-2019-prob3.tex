Let $E$ be the intersection point of the angle bisectors of $\angle DAB$ and $\angle CXB$. We will show that $E$ lies on line $BC$.

Since $\omega$ is tangent to $AC$ at $A$ and passes through $I$, we have $AI$ as the angle bisector of $\angle BAC$. Let $F$ be the intersection point of $AI$ and $BC$. Since $AI$ is the angle bisector of $\angle BAC$, we have $\angle BAF = \angle CAF$. Also, since $AD$ is the angle bisector of $\angle BAC$, we have $\angle BAD = \angle CAD$. Thus, $\angle BAF = \angle CAD$, which implies that $AFCD$ is cyclic.

Now, let $Y$ be the intersection point of $BC$ and the angle bisector of $\angle CXB$. Since $AFCD$ is cyclic, we have $\angle AFD = \angle ACD = \angle XCB$. Also, since $AI$ is the angle bisector of $\angle BAC$, we have $\angle AID = \angle BAI = \angle CAX$. Thus, $\angle AFD = \angle AID$, which implies that $AFDI$ is cyclic.

Now, let $Z$ be the intersection point of $BC$ and the angle bisector of $\angle DAB$. Since $AFDI$ is cyclic, we have $\angle AID = \angle AFD = \angle AZD$. Also, since $AD$ is the angle bisector of $\angle BAC$, we have $\angle AZD = \angle DAB$. Thus, $\angle DAB = \angle AID$, which implies that $ADIZ$ is cyclic.

Therefore, $Z$ is the intersection point of the angle bisectors of $\angle DAB$ and $\angle CXB$, which lies on line $BC$. This completes the proof.
