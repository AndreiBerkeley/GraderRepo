Let $r$ be the number of red beads and $b$ be the number of blue beads in the necklace. Since there are $mn$ beads in total, we have $r+b=mn$.

Given that each block of $n$ consecutive beads has a distinct number of red beads, we can consider the number of red beads in each block. Since there are $m$ blocks, there are $m$ distinct numbers of red beads. The smallest possible number of red beads in a block is $0$, and the largest possible number of red beads in a block is $n-1$.

Therefore, the $m$ distinct numbers of red beads in the blocks are $0, 1, 2, \ldots, n-1$. The sum of these $m$ distinct numbers must be less than or equal to the total number of red beads $r$ in the necklace. The sum of the first $n$ positive integers is $\frac{n(n-1)}{2}$, so we have:
$$\frac{n(n-1)}{2} \leq r$$

Since $r+b=mn$, we have $r=mn-b$. Substituting this into the inequality above, we get:
$$\frac{n(n-1)}{2} \leq mn-b$$
$$n(n-1) \leq 2mn-2b$$
$$n^2-n \leq 2mn-2b$$
$$n^2-(2m)n+n \leq -2b$$

Since $n$ is a positive integer, the discriminant of this quadratic inequality must be non-positive:
$$(2m)^2-4(1)(1) \geq 0$$
$$4m^2-4 \geq 0$$
$$m^2 \geq 1$$
$$m \geq 1$$

Therefore, the possible values of $m$ are $1, 2, 3, \ldots$.

Now, let's consider the possible values of $n$ for each $m$.

For $m=1$, the inequality simplifies to $n \leq -2b$, which is not possible since $n$ is a positive integer. Therefore, $m=1$ is not a valid solution.

For $m=2$, the inequality simplifies to $n^2-4n+1 \leq -2b$. Since $n$ is a positive integer, we have $n=1$ as the only possible solution. Checking, we see that $n=1$ satisfies the conditions.

For $m=3$, the inequality simplifies to $n^2-6n+1 \leq -2b$. The only possible solution for $n$ is $n=1$. Checking, we see that $n=1$ satisfies the conditions.

For $m=4$, the inequality simplifies to $n^2-8n+1 \leq -2b$. The only possible solution for $n$ is $n=1$. Checking, we see that $n=1$ satisfies the conditions.

For $m=5$, the inequality simplifies to $n^2-10n+1 \leq -2b$. The only possible solution for $n$ is $n=1$. Checking, we see that $n=1$ satisfies the conditions.

Continuing this pattern, we see that the only valid solution is $(m,n) = (2,1)$.

Therefore, the only possible values of the ordered pair $(m, n)$ are $(2, 1)$.
