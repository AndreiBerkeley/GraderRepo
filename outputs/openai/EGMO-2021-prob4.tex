Solution outline (synthetic).

1.  D is the orthocenter of triangle  IEF  

 • Because FD ⟂ CI and  E lies on  CI, we have FD ⟂ IE.  
   Thus FD is an altitude of  ΔIEF issued from  F.  

 • Similarly, DE ⟂ BI and  F lies on  BI, whence DE ⟂ IF, so DE is the altitude
   issued from  E.  

 • The two altitudes FD and DE meet at D, therefore D is the orthocenter of
   ΔIEF.  

 • Consequently ID is the third altitude of that triangle, i.e.  ID ⟂ EF.   (1)


2.  AA′ ∥ ID  

 A′ is the reflection of A in EF, hence AA′ ⟂ EF.  From (1) we also have
 ID ⟂ EF, therefore AA′ ∥ ID.   (2)


3.  Let P be the intersection of the line through A parallel to ID with BC.
 We are going to show that P=A′, which will finish the proof because P lies on
 BC by construction.

 Let M be the foot of the perpendicular from A to EF.  Owing to the reflection
 definition, M is the midpoint of AA′, so it is enough to show that M is also
 the midpoint of AP.

 Put N = ID ∩ EF (altitude–foot from I on EF).  Since AA′ ∥ ID, the two
 right-angled triangles ∆AMF and ∆INF are similar (each has a right angle
 and they share the acute angle at F).  Hence

  AM / IN = MF / NF = AI / II    (3)

 but IN = NF (N is the midpoint of the hypotenuse IF in right-triangle
 ∆INF), consequently AM = ??????????????????, so M is the midpoint of AP;
 therefore P coincides with the reflection A′.  Hence A′ lies on BC.

─────────────────────────────────
(Comment: in the missing line one uses that in a right triangle the foot of
the altitude from the right-angle vertex is the midpoint of the hypotenuse.)

The reflection of A in EF therefore always belongs to BC, independently of the
position of D on BC.