Let $D$ be the point where the incircle of $\triangle ABC$ touches side $BC$. Let $E$ be the intersection point of $AI$ with the incircle. Let $F$ be the intersection point of $PQ$ with $BC$.

Since $I$ is the incenter of $\triangle ABC$, we have $\angle BID = \angle A/2$ and $\angle CID = \angle A/2$. Since $IB$ is tangent to the circle through $B$ tangent to $AI$ at $I$, we have $\angle IBD = \angle BID = \angle A/2$. Similarly, $\angle ICD = \angle A/2$. Therefore, $BD = CD$.

Since $IE$ is the angle bisector of $\angle BIC$, we have $\angle BIE = \angle CIE = 90^\circ + \angle A/2$. Since $IE$ is also the angle bisector of $\angle BIC$, we have $\angle BIE = \angle CIE = 90^\circ + \angle A/2$. Therefore, $BICE$ is cyclic.

Now, we have $\angle BPE = \angle BIE = 90^\circ + \angle A/2$ and $\angle CQI = \angle CIE = 90^\circ + \angle A/2$. Since $BICE$ is cyclic, we have $\angle BIE = \angle CIE$. Therefore, $\angle BPE = \angle CQI$. This implies that $BP \parallel CI$ and $CQ \parallel BI$.

Since $BP \parallel CI$ and $CQ \parallel BI$, we have $\angle BPI = \angle ICB = \angle IBC = \angle QCI$. Therefore, $\angle BPI = \angle QCI$. This implies that $BP = CI$ and $CQ = BI$.

Since $BD = CD$ and $BP = CI$, we have $FP = FD$. Similarly, $FQ = FD$. Therefore, $FP = FQ$, which implies that $PQ$ is tangent to the incircle of $\triangle ABC$ at $F$.
