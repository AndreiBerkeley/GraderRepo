Let's first count the number of integer sequences \( x_1,\dots,x_k,y_1,\dots,y_m,z \) satisfying the given conditions.

For a fixed value of \( z \), the number of ways to choose \( x_1,\dots,x_k \) and \( y_1,\dots,y_m \) is given by the number of ways to choose \( k \) elements from the set \( \{1,2,\dots,z\} \) and \( m \) elements from the set \( \{1,2,\dots,z\} \), respectively. This can be calculated using binomial coefficients as \( \binom{z}{k} \) and \( \binom{z}{m} \), respectively.

Since \( z \) can take values from 1 to \( n \), the total number of sequences is the sum of the number of sequences for each value of \( z \), i.e.,

\[
f(n) = \sum_{z=1}^{n} \binom{z}{k} \binom{z}{m}.
\]

Now, we will show that the expression \( f(n) \) can be expressed as a polynomial in \( n \) with nonnegative coefficients.

Expanding the binomial coefficients, we have

\[
\binom{z}{k} = \frac{z(z-1)\cdots(z-k+1)}{k!} \quad \text{and} \quad \binom{z}{m} = \frac{z(z-1)\cdots(z-m+1)}{m!}.
\]

Multiplying these two expressions and summing over all \( z \) from 1 to \( n \), we get

\[
f(n) = \sum_{z=1}^{n} \frac{z(z-1)\cdots(z-k+1)}{k!} \cdot \frac{z(z-1)\cdots(z-m+1)}{m!}.
\]

Now, we can rewrite this expression as a polynomial in \( n \) with nonnegative coefficients by expanding the product and summing over all terms. This can be done by expanding the product of the two fractions and then summing the terms with the same power of \( n \). The resulting expression will be a polynomial in \( n \) with nonnegative coefficients, which completes the proof.
