Yes, it does.  
In fact, once the number \(2021\) is declared to be fantabulous, the two rules that are given force every positive integer to be fantabulous, and therefore in particular the number  
\[
2021^{2021}
\]
is fantabulous.

--------------------------------------------------------------------
Proof (by contradiction).

Say that two positive integers are *linked* if one can be obtained from the other by repeatedly replacing an integer \(m\) by one of the two integers \(2m+1\) or \(3m\), or by carrying out the inverse steps
\[
m\longleftarrow 3m,\qquad  
m\longleftarrow 2m+1
\ (\hbox{that is }m\mapsto \tfrac{m-1}{2},\hbox{ allowed whenever }m\text{ is odd}).
\]
Because of the wording of the statement, “if **one** of
\(\{m,2m+1,3m\}\) is fantabulous, then **all three** are”, a set of
fantabulous numbers can be *any* union of such linked‐components.  
Hence a single positive integer \(N\) will automatically drag with it **every** integer that is linked to \(N\).

We therefore show that every positive integer is linked to \(2021\).
Assume, to the contrary, that this is false and choose the smallest
positive integer

\[
n\;>\;1
\]

that is **not** linked to \(2021\).
(Equivalently, \(n\) is the smallest positive integer that cannot be
reached from \(2021\) by the allowed moves.)

--------------------------------------------------------------------
Step 1 –  odd numbers are harmless.

If \(n\) is odd, set \(m=\dfrac{n-1}{2}\;(>0)\).
Then \(n=2m+1\) so that \(n\) and \(m\) lie in the same triple
\(\{m,2m+1,3m\}\); hence \(n\) is linked to \(m\).
Because \(m<n\), \(m\) **is** linked to \(2021\) by the minimality of
\(n\); consequently \(n\) is linked to \(2021\) as well – contradiction.
Thus our non-linked integer \(n\) cannot be odd; it must be even.

--------------------------------------------------------------------
Step 2 –  multiples of \(3\) are harmless.

If \(3\mid n\) write \(n=3m\;(>0)\).
Again \(m<n\) and \(n,\,m\) belong to the same triple
\(\{m,2m+1,3m\}\); exactly as before this forces \(n\) to be linked to
\(2021\), contradicting the choice of \(n\).
Hence \(n\) is *not* a multiple of \(3\).

--------------------------------------------------------------------
Step 3 –  the only remaining shape.

So far we know that  
 • \(n\) is even,  
 • \(3\nmid n\).  

Every even integer is congruent to \(0,2,4\pmod 6\); discarding the
case \(0\pmod 3\) leaves the two possibilities

\[
n\equiv 2\pmod 6\quad\text{or}\quad n\equiv 4\pmod 6 .
\]

--------------------------------------------------------------------
Step 4 –  the case \(n\equiv 4\bmod 6\) also collapses.

Write \(n=6k+4 \;(k\ge 0)\).
Form

\[
p=2n+1=2(6k+4)+1=12k+9=3(4k+3).
\]
The three integers
\(\bigl\{n,\,p,\,3n\bigr\}\) constitute one triple; moreover
\(p/3=4k+3<n\).
By the minimality of \(n\) the smaller number \(4k+3\) is linked to
\(2021\), whence \(p\) is linked to \(2021\), and finally \(n\) as well.
This contradicts the choice of \(n\).
Therefore the residue class \(4\pmod 6\) cannot occur; the only option
left is

\[
n\equiv 2\pmod 6 .
\]

--------------------------------------------------------------------
Step 5 –  the *last* possible shape also yields a contradiction.

Write

\[
n=6k+2\qquad(k\ge 0,\;n>2).
\]

(Notice that \(n=2\) itself cannot be our counter-example, because
\(2\to 6\to13\to27\to9\to3\to1\) is a chain of allowed moves,
explicitly linking \(2\) to \(1\), hence to \(2021\).)

Now carry out the following four allowed moves:

\[
\begin{aligned}
n&\longrightarrow 3n=18k+6,\\[2pt]
18k+6&\longrightarrow 2(18k+6)+1=36k+13,\\[2pt]
36k+13&\longrightarrow 2(36k+13)+1=72k+27
       \;=\;3(24k+9),\\[2pt]
72k+27&\longrightarrow\tfrac{72k+27}{3}=24k+9.
\end{aligned}
\]

Set \(n' = 24k+9\).  A short calculation gives

\[
n' - n = (24k+9)-(6k+2)=18k+7>0,
\qquad
n' < 3n = 18k+6 < 3n + 3 .
\]

In particular \(n' < 3n\); combining with \(n=6k+2>2\) one has
\(
n'<n\;.
\)

We have just produced a chain of allowed moves that links \(n\) to a
smaller positive integer \(n'\).  
But by the minimal choice of \(n\), that smaller integer *is* linked to
\(2021\); hence so is \(n\), contradicting once more the assumption that
\(n\) was not linked.

--------------------------------------------------------------------
Conclusion.

Every attempt to exhibit a positive integer that is *not* linked to
\(2021\) has failed.  
Therefore every positive integer is linked to \(2021\); in other words
the single assumption “\(2021\) is fantabulous’’ forces **all**
positive integers to be fantabulous.  

In particular the number \(2021^{2021}\) is fantabulous. ∎