Answer  
All the integers \(k\) that work are the even ones,
\[ k=2,4,6,\dots \]

Explanation  

1.  Odd exponents do ​not​ work  
   ------------------------------------------------  
   Take \(n=2\).  Then  
   \[
       \sum_{i=0}^{2}\binom 2i^{\,k}=1+2^{\,k}+1=2^{\,k}+2 .
   \]
   One has \(n+1=3\).  
   Because \(2\equiv-1\pmod 3\),  
   \[
        2^{\,k}+2\equiv(-1)^{\,k}-1\pmod 3 .
   \]
   When \(k\) is odd this is
   \((-1)-1\equiv1\pmod3\neq0\);
   hence the sum is not divisible by \(3\).
   Therefore every admissible \(k\) must be even.

2.  Every even exponent works  
   -----------------------------------------  
   Write an even exponent as \(k=2m\;(m\ge1)\).
   Fix a positive integer \(n\) and put \(N=n+1\).
   Let  
   \[
        \omega=e^{2\pi i/N}
   \]
   be a primitive \(N\)-th root of unity and put
   \[
        f(x)=(1+x)^{\,n}.
   \]

   ``The constant–term trick’’. –  
   Look at the polynomial
   \[
        P(x)=\Bigl(f(x)\,f(x^{-1})\Bigr)^{m}
             =\Bigl((1+x)^{\,n}(1+x^{-1})^{\,n}\Bigr)^{m}.
   \]
   Expanding \(P(x)\) shows that its constant term is exactly the
   required power–sum:
   \[
        \operatorname{ct}\, P(x)=\sum_{i=0}^{n}\binom ni^{2m},
        \qquad (\ast)
   \]
   because a term
   \(\binom ni^{m}\binom nj^{m}\,x^{\,i-j}\)
   coming from each of the
   \(m\) identical factors contributes to the constant term
   only when all the chosen exponents coincide, i.e. when \(i=j\).

   ``Extracting the constant term with a root–of–unity filter’’. –  
   For every polynomial \(Q(x)\) with complex coefficients one has
   \[
        \operatorname{ct}\,Q(x)=\frac1N
             \sum_{t=0}^{N-1}Q(\omega^{t}),
        \qquad(\dagger)
   \]
   because \(\sum_{t=0}^{N-1}\omega^{tr}=N\) if \(N\mid r\) and \(0\)
   otherwise.  Applying (\(\dagger\)) to \(P\) and using (\(\ast\)) gives
   \[
        \boxed{\;
        \sum_{i=0}^{n}\binom ni^{2m}
        =\frac1N\sum_{t=0}^{N-1}
                \Bigl((1+\omega^{t})^{\,n}(1+\omega^{-t})^{\,n}\Bigr)^{m}
        } .
   \]

   For each \(t\) the factor
   \( (1+\omega^{t})^{\,n}(1+\omega^{-t})^{\,n}=
      |1+\omega^{t}|^{\,2n}\)
   is a non–negative real number;
   raising it to the power \(m\) keeps it real.
   Thus every term on the right–hand side is real,
   and their sum is an integer (because it equals the integer on
   the left–hand side).  Being an integer, that sum is congruent to
   \(0\pmod N\) if and only if it is a multiple of \(N\).

   But the left–hand side is the very sum we are studying; writing
   it again as
   \[
        \sum_{i=0}^{n}\binom ni^{2m}
        =\frac1{N}\Bigl(\text{integer divisible by }N\Bigr)
   \]
   shows that it is ​itself​ a multiple of \(N\,(=n+1)\).

   Consequently \(n+1\;\bigl|\;\sum_{i=0}^{n}\binom ni^{2m}\) for every
   \(n\) whenever \(k=2m\) is even.

3.  Conclusion  
   ----------------  
   Odd \(k\) fail (step 1) and every even \(k\) works (step 2).  Hence
   all and only the even positive integers \(k\) satisfy the required
   divisibility.