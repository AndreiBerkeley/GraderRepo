Let  

S_n(α)=⌊α⌋+⌊2α⌋+⋯+⌊nα⌋ ,  n=1,2,…

We look for all real α such that n | S_n(α) for every positive integer n.

Notation  
{x}=x−⌊x⌋ is the fractional part of x, 0≤{x}<1.  
Write α=t+β with t=⌊α⌋∈ℤ and β∈[0,1).

-------------------------------------------------
1.  Even integers work
-------------------------------------------------
For an even integer α=2k we have  

S_n(2k)=2k(1+2+⋯+n)=2k· n(n+1) 2 =k n(n+1)=n·k(n+1),

hence n divides S_n(2k) for every n.  
Thus every even integer belongs to the required set.

-------------------------------------------------
2.  A convenient reformulation
-------------------------------------------------
Because {kα}={kβ},  

S_n(α)=∑_{k=1}^{n}(kα−{kα})  
      =α n(n+1) 2 −∑_{k=1}^{n}{kβ}.                         (1)

Divide (1) by n :

T_n:= S_n(α) n =α(n+1) 2 −u_n,             (2)

where  

u_n:= 1 n ∑_{k=1}^{n}{kβ}∈[0,1).          (3)

The hypothesis “n | S_n(α)” is equivalent to  

T_n∈ℤ for every n.                         (4)

-------------------------------------------------
3.  How fast can u_n vary?
-------------------------------------------------
From (3):

u_{n+1}= 1 n+1 (n u_n+{(n+1)β}).

Hence  

|u_{n+1}−u_n|= |{(n+1)β}−u_n| n+1 ≤ 1 n+1 .         (5)

So the sequence (u_n) changes extremely slowly: the jump from n to n+1
is at most 1/(n+1).

-------------------------------------------------
4.  Consequences if β≠0
-------------------------------------------------
Assume β≠0.  
From (2),

fractional part of T_n = fractional part of ( β(n+1) 2 −u_n).         (6)

Denote  

f_n:= β(n+1) 2 −u_n  (mod 1).

Because of (4), f_n must always cancel the fractional part contributed
by t(n+1)/2.

There are two possibilities.

A.  t is even  
Then t(n+1)/2 is integral, so (4) demands f_n≡0 (mod 1) for all n.

B.  t is odd  
Then t(n+1)/2 is integral if n is odd and has fractional part ½ if n
is even; consequently

    f_n≡0 (n odd),  f_n≡½ (n even).        (7)

-------------------------------------------------
5.  But f_n cannot oscillate so little
-------------------------------------------------
Compute the difference of successive f_n’s:

f_{n+1}−f_n = β 2 −(u_{n+1}−u_n).             (8)

Using the bound (5):

|f_{n+1}−f_n−β/2| ≤ 1 n+1 .                   (9)

Because β≠0, the constant |β|/2 is positive; choose N so large that
|β|/2−1/(N+1)>0.  
For all n≥N,

|f_{n+1}−f_n| ≥ |β| 2 − 1 n+1 ≥ |β| 2 − 1 N+1 >0.             (10)

Thus (f_n) must move by at least a fixed positive amount for every large
n, which is impossible in either situation A or B:

• In A we would have f_n constantly 0, so every difference is 0.  
• In B the sequence alternates between two fixed values (0 and ½),
hence |f_{n+1}−f_n| is either 0 or ½, contradicting (10) once n is
large enough (½ cannot equal the small bound in (10) for large n).

Therefore β must be 0.

-------------------------------------------------
6.  So α is an integer – what kind?
-------------------------------------------------
Put β=0 in (4).  Then

S_2(α)=⌊α⌋+⌊2α⌋=α+2α=3α.

The condition 2 | 3α forces α to be even (because gcd(2,3)=1).  
Hence α=2k, k∈ℤ.

-------------------------------------------------
7.  Conclusion
-------------------------------------------------
The only real numbers satisfying the requirement are the even
integers.

Answer.  α ∈ {…, −4, −2, 0, 2, 4, …} = 2ℤ.