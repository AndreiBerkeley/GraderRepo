Let $D$ be the foot of the altitude from $A$ to $BC$.
Since $F$ is the foot of the altitude from $C$ to $AB$, we have $\angle AFC = 90^\circ$.
Since $P$ is the reflection of $H$ across $BC$, we have $HP = HC$.
Since $H$ is the orthocenter of triangle $ABC$, we have $AH \perp BC$ and $CH \perp AB$.
Therefore, $\angle AHC = 90^\circ$.
Since $P$ is the reflection of $H$ across $BC$, we have $\angle AHP = \angle AHC = 90^\circ$.
Therefore, $A$, $H$, $P$, and $C$ are concyclic.
Since $A$, $H$, $P$, and $C$ are concyclic, we have $\angle APC = \angle AHC = 90^\circ$.
Since $A$, $F$, $P$, and $C$ are concyclic, we have $\angle APC = \angle AFC = 90^\circ$.
Therefore, $A$, $H$, $P$, $F$, and $C$ are concyclic.
Since $A$, $H$, $P$, $F$, and $C$ are concyclic, we have $\angle APF = \angle ACF = 90^\circ$.
Since $A$, $H$, $P$, $F$, and $C$ are concyclic, we have $\angle AHP = \angle AFP$.
Since $A$, $H$, $P$, $F$, and $C$ are concyclic, we have $\angle ACP = \angle AFP$.
Therefore, $\angle ACP = \angle AHP$.
Since $A$, $H$, $P$, and $C$ are concyclic, we have $\angle ACP = \angle AHP = 90^\circ$.
Therefore, $CP$ is a diameter of the circumcircle of triangle $AFP$.
Since $X$ and $Y$ are the intersection points of the circumcircle of triangle $AFP$ with line $BC$, we have $\angle AXP = \angle AYP = 90^\circ$.
Therefore, $X$ and $Y$ lie on the circle with diameter $CP$.
Since $CP$ is a diameter of the circle passing through $X$ and $Y$, we have $CX = CY$.
