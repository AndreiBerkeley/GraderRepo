Let $X$ be the reflection of $A$ across the line $EF$. We will show that $X$ lies on line $BC$.

Since $EF$ is perpendicular to $BI$, we have $\angle BFE = 90^\circ$. Similarly, since $EF$ is perpendicular to $CI$, we have $\angle CEF = 90^\circ$. Therefore, $BFEI$ and $CEFI$ are cyclic quadrilaterals.

Now, we have $\angle BFI = \angle BEI = \angle BAI$ and $\angle CFI = \angle CEI = \angle CAI$. Since $X$ is the reflection of $A$ across $EF$, we have $\angle BFX = \angle BAX$ and $\angle CEX = \angle CAX$. Thus, $\angle BFX + \angle CEX = \angle BAX + \angle CAX = \angle BAC = 180^\circ - \angle BCA$. This implies that $BFXE$ is a cyclic quadrilateral.

Therefore, $\angle EXF = \angle EBF = \angle EBI = \angle IBC$. Similarly, $\angle FXE = \angle ICB$. Since $\angle EXF + \angle FXE = \angle IBC + \angle ICB = 180^\circ$, we have $X \in BC$.

Hence, the reflection of $A$ across the line $EF$ lies on the line $BC$.
