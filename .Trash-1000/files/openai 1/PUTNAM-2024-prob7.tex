Let's denote the number in the square in the \(i\)th row and \(j\)th column as \(a_{ij} = i+j-k\). We want to find \(n\) squares such that the numbers contained in them are exactly \(1,2,\dots,n\).

Let's consider the sum of the numbers in the selected squares. Since we want the numbers to be \(1,2,\dots,n\), the sum of these numbers is \(\frac{n(n+1)}{2}\).

The sum of the numbers in the \(n\) selected squares can be expressed as:
\[\sum_{i=1}^{n} \sum_{j=1}^{n} a_{ij} = \sum_{i=1}^{n} \sum_{j=1}^{n} (i+j-k) = \sum_{i=1}^{n} \left( \sum_{j=1}^{n} i + \sum_{j=1}^{n} j - \sum_{j=1}^{n} k \right).\]

Solving the above expression, we get:
\[\frac{n(n+1)(2n+1)}{6} + n(n+1) - nk = \frac{n(n+1)}{2}.\]

Simplifying, we get:
\[n^2 + 3n - 6k = 0.\]

This is a quadratic equation in \(n\). For the equation to have a positive integer solution for \(n\), the discriminant must be a perfect square. Thus, we have:
\[9 + 24k = m^2\]
for some positive integer \(m\).

Rearranging, we get:
\[24k = m^2 - 9 = (m+3)(m-3).\]

Since \(k\) is a positive integer, the possible values for \(m\) are \(4, 6, 8, 12, 24\). Substituting these values back into the equation, we find the corresponding values for \(k\).

Therefore, the possible values for \(n\) and \(k\) are:
\[(n,k) = (1,0), (3,1), (6,3), (15,9), (276,81).\]
