Let $M$ be the midpoint of $BC$. Since $S$ is the midpoint of the arc $CB$ of $\Omega$, we have $SM \perp BC$. Let $T$ be the intersection of $SM$ and $AE$. Since $SM \parallel AD$, we have $\angle AST = \angle ADE = 90^\circ$. This implies that $ASTD$ is cyclic.

Since $ASTD$ is cyclic, we have $\angle STD = \angle SAD = \angle SAB$. Since $SM$ is the perpendicular bisector of $BC$, we have $BM = CM$. Thus, $\angle SAB = \angle SBM$. Therefore, $\angle STD = \angle SBM$.

Since $AD \parallel ML$, we have $\angle LMD = \angle SAD = \angle SBM$. This implies that $MDLB$ is cyclic.

Let $Q$ be the intersection of $BL$ and $AC$. Since $MDLB$ is cyclic, we have $\angle BMD = \angle BLD$. Since $MD \parallel AC$, we have $\angle BMD = \angle BAC$. Therefore, $\angle BLD = \angle BAC$.

Since $MDLB$ is cyclic, we have $\angle BLD = \angle BMD = \angle BAC$. This implies that $BL$ is the angle bisector of $\angle BAC$.

Now, let $N$ be the intersection of $BS$ and the tangent to $\omega$ at $P$. Since $P$ lies on $\omega$, we have $\angle BPL = \angle BDL$. Since $MDLB$ is cyclic, we have $\angle BDL = \angle BML$. Since $SM \parallel AD$, we have $\angle BML = \angle BAC$. Therefore, $\angle BPL = \angle BAC$.

Since $BL$ is the angle bisector of $\angle BAC$, we have $NL$ bisects $\angle BNS$. Since $N$ lies on the tangent to $\omega$ at $P$, we have $NP$ is also tangent to $\omega$. Therefore, the line tangent to $\omega$ at $P$ meets line $BS$ on the internal angle bisector of $\angle BAC$.
