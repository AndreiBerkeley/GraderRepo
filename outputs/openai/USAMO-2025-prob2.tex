We argue by contradiction.  
Throughout, “root’’ always means a real root.

Assume that ​P has only real roots.  
Because the roots are simple and \(P(0)\neq 0\) we may write  

\[
P(x)=c\,\prod_{i=1}^{n}(x-r_i),\qquad c\in\mathbb R\setminus\{0\},
\qquad r_1,\dots ,r_n\in\mathbb R\setminus\{0\},\;
r_i\neq r_j\ (i\neq j).
\tag{1}
\]

For a finite set \(S\subset\{1,\dots ,n\}\) put  

\[
e_m(S)=\!\!\sum_{i_1<\dots <i_m\in S}\!\! r_{i_1}\dots r_{i_m}\qquad(0\le m\le |S|),
\]
(the elementary symmetric sums; by convention \(e_0(S)=1\)).
If \(S=\{i_1,\dots ,i_s\}\) we abbreviate  

\[
Q_S(x)=\prod_{i\in S}(x-r_i)=x^s- e_1(S)x^{s-1}+e_2(S)x^{s-2}-\dots +(-1)^s e_s(S).
\tag{2}
\]

Thus the coefficient of \(x^{s-m}\;(1\le m\le s )\) equals \((-1)^m e_m(S)\).

The aim is to exhibit a \(k\)-element subset \(S\) for which  
\(e_1(S),e_2(S),\dots ,e_k(S)\) are all non–zero; then \(Q_S\) divides \(P\)
and every coefficient of \(Q_S\) is non–zero, contradicting the hypothesis.

--------------------------------------------------------------------
Step 1.  If at least \(k\) roots have the same sign we are done.

Indeed, if \(r_{i_1},\dots ,r_{i_k}\) are all positive (or all negative) then every
\(e_m(S)\;(1\le m\le k)\) is strictly positive (respectively strictly negative),
hence non–zero.

So from now on we may (and do) suppose

\[
\text{there are fewer than }k\text{ positive roots and fewer than }k\text{ negative roots.}
\tag{3}
\]

Because \(n>k\) we necessarily have  

\[
n\le 2k-2\qquad\text{and hence }n\ge k+1.
\tag{4}
\]

--------------------------------------------------------------------
Step 2.  A property of real–rooted polynomials.

Lemma.  
Let \(d\ge 2\) and let \(R=\{s_1,\dots ,s_d\}\subset\mathbb R\setminus\{0\}\)
be a set of \(d\) distinct real numbers.  
For its elementary symmetric sums \(E_m=e_m(R)\) one never has two
consecutive zeros:
\[
E_{m-1}=E_{m}=0\qquad(1\le m\le d-1)\quad\text{is impossible.}
\tag{5}
\]

Proof.  
Put \(F(x)=\prod_{t\in R}(x-t)=x^d+\sum_{j=0}^{d-1}c_jx^{\,j}\)
with \(c_{d-m}=(-1)^mE_m\).
If (5) held, the coefficients of \(x^{d-m}\) and \(x^{d-m-1}\) in \(F\)
would be zero, so  

\[
F(x)=x^{d-m-2}(x^2+px+q),\quad p,q\in\mathbb R.
\]
Hence the two real numbers \(\alpha ,\beta\) that satisfy
\(x^2+px+q=0\) are roots of multiplicity at least \(d-m-1\ge 1\),
contradicting the fact that the roots of \(F\)
are simple. ∎

--------------------------------------------------------------------
Step 3.  Choosing \(k\) suitable roots.

Because of (4) we can pick an arbitrary set  
\(T=\{t_1,\dots ,t_{k+1}\}\subset\{r_1,\dots ,r_n\}\) with \(k+1\) elements.
Write \(E_m=e_m(T)\;(0\le m\le k+1)\).

For every index \(m\;(1\le m\le k-1)\) with \(E_{m-1}\neq 0\) define the single
real number  

\[
\lambda_m=\frac{E_m}{E_{m-1}} .
\tag{6}
\]

Because of the lemma the indices with \(E_{m-1}=0\) automatically satisfy
\(E_m\neq 0\); thus (6) produces at most \(k-1\) distinct real numbers.
Since \(T\) itself contains \(k+1\) distinct numbers, there exists  

\[
t\in T\quad\text{such that }t\neq\lambda_m\ \text{ for every }m\text{ with }E_{m-1}\neq 0.
\tag{7}
\]

Put \(S=T\setminus\{t\}\); hence \(|S|=k\).

--------------------------------------------------------------------
Step 4.  The coefficients corresponding to \(S\) are non–zero.

For \(1\le m\le k-1\) we have  

\[
e_m(S)=E_m-tE_{m-1}.
\tag{8}
\]

•  If \(E_{m-1}\neq 0\), (7) gives \(t\neq\lambda_m\); thus \(e_m(S)\neq 0\).

•  If \(E_{m-1}=0\), the lemma tells us \(E_m\neq 0\); then (8) reduces to
\(e_m(S)=E_m\neq 0\).

Finally \(e_k(S)=(-1)^k\!\!\!\!\prod_{i\in S}\!\!r_i\neq 0\) because none of the
roots is \(0\).
Therefore every elementary symmetric sum
\(e_1(S),\dots ,e_k(S)\) is non–zero, and by (2) every
coefficient of \(Q_S\) is non–zero.

--------------------------------------------------------------------
Step 5.  Conclusion.

The polynomial \(Q_S\) has degree \(k\), all its coefficients are non–zero,
and, by construction, \(Q_S\mid P\).  
This contradicts the hypothesis of the problem.
Hence our initial assumption that all roots of \(P\) are real is impossible;  
therefore \(P\) possesses at least one non–real root. ∎