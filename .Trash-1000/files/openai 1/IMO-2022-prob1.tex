Let's denote the sequence of coins as \( S = s_1s_2\ldots s_{2n} \), where \( s_i \) is either \( A \) or \( B \) for \( i = 1, 2, \ldots, 2n \). We will consider the following cases:

1. If \( S \) has no chain of length \( k \) or longer, then the process will not change the order of the coins. Thus, the leftmost \( n \) coins will all be of the same type if and only if the first \( n \) coins are all of the same type. This implies that \( s_1 = s_2 = \ldots = s_n \).

2. If \( S \) has a chain of length \( k \) or longer, then the process will move all coins in that chain to the left end of the row. Let \( m \) be the length of the longest chain containing the \( k \)th coin from the left. Then, the process will change the order of the coins such that the first \( m \) coins are all of the same type. This implies that \( s_1 = s_2 = \ldots = s_m \).

Now, let's analyze the possible values of \( m \) for a given \( k \):

- If \( k \leq n \), then the longest chain containing the \( k \)th coin from the left is of length \( k \). Thus, \( m = k \).
- If \( n < k \leq 2n \), then the longest chain containing the \( k \)th coin from the left is of length \( 2n - k + 1 \). Thus, \( m = 2n - k + 1 \).

Therefore, the leftmost \( n \) coins will all be of the same type if and only if \( s_1 = s_2 = \ldots = s_m \), where \( m = \min(k, 2n - k + 1) \).

Hence, the pairs \( (n, k) \) such that for every initial ordering, at some moment during the process, the leftmost \( n \) coins will all be of the same type are given by:
\[ (n, k) = \{(n, k) \mid 1 \leq k \leq n \text{ or } n < k \leq 2n\} \]
