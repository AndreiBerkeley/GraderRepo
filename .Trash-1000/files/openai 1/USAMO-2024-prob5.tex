Let \( \angle BAC = \alpha \) and \( \angle ACB = \beta \). Since \( \angle DAC = \angle ACB = \beta \), we have \( \angle DAB = \angle DAC - \angle BAC = \beta - \alpha \). Similarly, since \( \angle BDC = 90^\circ + \angle BAC = 90^\circ + \alpha \), we have \( \angle BDE = \angle BDC - \angle EDC = 90^\circ + \alpha - \beta \).

Since \( AE = EC \), we have \( \angle EAC = \angle ECA = \frac{1}{2} \angle ACE = \frac{1}{2} \beta \). Therefore, \( \angle EAB = \angle EAC - \angle BAC = \frac{1}{2} \beta - \alpha \).

Now, consider the quadrilateral \( ABDE \). We have \( \angle ABE = \angle ABD + \angle DBE = \angle DAB + \angle BDE = (\beta - \alpha) + (90^\circ + \alpha - \beta) = 90^\circ \). This implies that \( AB \) is perpendicular to \( DE \) at point \( B \).

Let \( O \) be the circumcenter of triangle \( BEM \). Since \( M \) is the midpoint of \( BC \), we have \( OM \perp BC \), which implies that \( OM \parallel AB \). Since \( AB \perp DE \) at \( B \), we have \( OM \perp DE \). Therefore, \( O \) lies on the perpendicular bisector of \( DE \), which means that \( AB \) is tangent to the circumcircle of triangle \( BEM \) at point \( B \).
