Let $D_n$ be the largest number of dominoes that can be placed on a $2n \times 2n$ board satisfying the given conditions. We will prove that $D_n = 2n^2 - n$.

Consider the $2n \times 2n$ board. Color the cells of the board in an alternating pattern of black and white, like a chessboard. Since each domino covers two adjacent cells, one black and one white, the number of dominoes placed on the board must be half the number of cells on the board.

Let $B_n$ be the number of black cells and $W_n$ be the number of white cells on the $2n \times 2n$ board. Since the board has $2n \times 2n = 4n^2$ cells, we have $B_n + W_n = 4n^2$.

Now, consider the black cells. Each black cell has $2$ white cells adjacent to it. Since each domino covers one black and one white cell, each black cell contributes to the placement of $1$ domino. Therefore, the number of dominoes placed on the board is equal to the number of black cells, which is $B_n$.

Since each white cell is adjacent to exactly one black cell, the number of white cells is equal to the number of black cells. Thus, $W_n = B_n$.

Substitute $W_n = B_n$ into $B_n + W_n = 4n^2$ to get $2B_n = 4n^2$, which implies $B_n = 2n^2$.

Therefore, the largest number of dominoes that can be placed on a $2n \times 2n$ board is $D_n = B_n = 2n^2 = 2n^2 - n + n = 2n^2 - n$. 

Hence, $D_n = 2n^2 - n$ for all positive integers $n$.
