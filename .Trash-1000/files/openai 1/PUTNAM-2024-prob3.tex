Let $A = \{1,2,3\}$, $B = \{1,2,\dots,2024\}$, and $C = \{1,2,\dots,6072\}$. We can count the number of elements in $S$ by considering the choices for $T(1,j)$, $T(2,j)$, and $T(3,j)$ for each $j \in B$.

For each $j \in B$, there are $\binom{6072}{3}$ ways to choose $T(1,j)$, $T(2,j)$, and $T(3,j)$ such that $T(1,j) < T(2,j) < T(3,j)$. Then, for $j \in B$, there are $2023$ choices for $T(1,j)$, $T(2,j)$, and $T(3,j)$ such that $T(i,j) < T(i,j+1)$ for all $i \in A$.

Therefore, the total number of elements in $S$ is $\binom{6072}{3} \cdot 2023^{2024}$.

Now, let's count the number of elements in $S$ such that $T(a,b) < T(c,d)$ for some $a,c \in A$ and $b,d \in B$. There are $3$ choices for $a$ and $c$, and $2024^2$ choices for $b$ and $d$. For each choice of $a$, $c$, $b$, and $d$, there are $2$ possible orderings for $T(a,b)$ and $T(c,d)$.

Therefore, the total number of elements in $S$ such that $T(a,b) < T(c,d)$ for some $a,c \in A$ and $b,d \in B$ is $3 \cdot 2024^2 \cdot 2$.

To find the fraction of elements in $S$ for which $T(a,b) < T(c,d)$ is at least $1/3$ and at most $2/3$, we need to find the range of values for this fraction. Let $N$ be the total number of elements in $S$ and $M$ be the number of elements in $S$ such that $T(a,b) < T(c,d)$ for some $a,c \in A$ and $b,d \in B$.

The fraction we are interested in is $\frac{M}{N} = \frac{3 \cdot 2024^2 \cdot 2}{\binom{6072}{3} \cdot 2023^{2024}}$.

We can simplify this expression to get $\frac{M}{N} = \frac{3 \cdot 2024^2 \cdot 2}{\frac{6072 \cdot 6071 \cdot 6070}{6} \cdot 2023^{2024}} = \frac{6 \cdot 2024^2}{6072 \cdot 6071 \cdot 2023^{2023}}$.

Since $2024^2 < 6072 \cdot 6071 \cdot 2023^{2023}$, we have $\frac{M}{N} < 1$. Therefore, there do not exist $a$ and $c$ in $A$ and $b$ and $d$ in $B$ such that the fraction of elements $T$ in $S$ for which $T(a,b) < T(c,d)$ is at least $1/3$ and at most $2/3$.
