Let  

                     n^k = a_0 + a_1(2n)+a_2(2n)^2+\dots + a_t(2n)^t ,
   
be the (ordinary) base–$2n$ expansion of $n^{k}$;  
thus $0\le a_i\le 2n-1$ and $t\le k$ (because $n^{k}<(2n)^k$).

Goal.  Show that for every fixed pair of positive integers $k,d$ there is
an odd integer $N=N(k,d)$ such that 

                       a_i > d   for every i ,         (1)

whenever $n\ge N$ is odd.

--------------------------------------------------------------------
1.  A convenient expression for the $i$-th digit
--------------------------------------------------------------------

For $i\ge 0$ write

                 q_i := ⌊ n^{k-i}/2^{\,i} ⌋ .                       (2)

Because dividing by $(2n)^i$ lowers the exponent of $n$ by $i$ and
introduces a factor $2^{-i}$, one has  

                 n^{k} / (2n)^i = n^{k-i}/2^{\,i} = q_i + β_i,
                 
where $0\le β_i<1$.  Consequently  

                 a_i ≡ q_i  (mod 2n) ,              0≤a_i≤2n-1.     (3)

So to obtain a **lower** bound for $a_i$ it is enough to bound $q_i$
from below, because when we reduce $q_i$ modulo $2n$ we never obtain a
negative number.

--------------------------------------------------------------------
2.  A lower bound for q_i
--------------------------------------------------------------------

Since $n$ is odd, $n^{k-i}$ is also odd, whence  

                 n^{k-i}= n·r                         (r odd).      (4)

Rewrite $r$ in base $2$ collecting exactly $i$ powers of $2$:

                 r = 2^{\,i} s + t ,           0 < t < 2^{\,i},  t odd.  (5)

Insert (5) into $n^{k-i}=n r$ and divide by $2^{\,i}$:

     n^{k-i}/2^{\,i} = n s + (n t)/2^{\,i}.                       (6)

Because $t<2^{\,i}$, the fraction $(n t)/2^{\,i}$ is **strictly less
than $n$**, and its integer part is

              b_i := ⌊ (n t)/2^{\,i} ⌋            with 0 ≤ b_i < n.  (7)

Equation (6) becomes

              q_i = n s + b_i .                                    (8)

Reducing (8) modulo $2n$ we keep at most one multiple of $n$:

       q_i  ≡                b_i            (mod 2n)  if s even,
       q_i  ≡     n +        b_i            (mod 2n)  if s odd.

Either way, since $0\le b_i<n$, we always have

                     a_i  ≥  b_i .                                  (9)

But from (7) and $t\ge 1$ we get the simple estimate

                     b_i  ≥  ⌊ n / 2^{\,i} ⌋ .                      (10)

Combining (9) and (10) we have proven

          a_i ≥ ⌊ n / 2^{\,i} ⌋      for every 0≤i≤t≤k.            (11)

--------------------------------------------------------------------
3.  Finishing the proof
--------------------------------------------------------------------

The right–hand side of (11) is **decreasing** in $i$.  Hence

      a_i ≥ ⌊ n / 2^{\,k} ⌋     for every digit a_i.               (12)

Choose  

                 N = 2^{\,k}(d+1)+1 ,   N odd.                     (13)

For every odd integer $n\ge N$ we have  

                 ⌊ n / 2^{\,k} ⌋  ≥  d+1 .                         (14)

Substituting (14) into (12) gives $a_i>d$ for every $i$, i.e. (1)
holds.  Therefore all digits of the base–$2n$ representation of
$n^{k}$ exceed the prescribed constant $d$ once $n$ is any odd integer
larger than $N(k,d)=2^{\,k}(d+1)+1$.

--------------------------------------------------------------------
4.  Conclusion
--------------------------------------------------------------------

For every fixed pair of positive integers $k$ and $d$ there exists an
odd integer $N=N(k,d)$ (explicitly, $N=2^{\,k}(d+1)+1$) such that for
every odd $n\ge N$ **each digit of $n^{k}$ written in base $2n$ is
strictly larger than $d$**.  This completes the proof.