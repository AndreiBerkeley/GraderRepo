Let $d = \gcd(a,b)$. Then we can write $a = da'$ and $b = db'$, where $\gcd(a',b') = 1$. 

Now, we have
\begin{align*}
\gcd(a^n+b, b^n+a) &= \gcd((da')^n + db', (db')^n + da') \\
&= \gcd(d(a'^n+b'), d(b'^n+a')) \\
&= \gcd(a'^n+b', b'^n+a').
\end{align*}

Thus, we can assume without loss of generality that $\gcd(a,b) = 1$. 

Now, let $d = \gcd(a^n+b, b^n+a)$. Then $d$ divides both $a^n+b$ and $b^n+a$, so it divides their sum and their difference:
\begin{align*}
d &\mid (a^n+b) + (b^n+a) = a^n + b^n + a + b, \\
d &\mid (a^n+b) - (b^n+a) = a^n - b^n + a - b.
\end{align*}

Adding and subtracting these two equations, we get
\begin{align*}
d &\mid 2(a^n + a), \\
d &\mid 2(a - b^n).
\end{align*}

Since $\gcd(a,b) = 1$, we have $\gcd(a,a-b) = 1$. Therefore, $d$ must divide $2$, which means $d = 1$ or $d = 2$. 

If $d = 1$, then $\gcd(a^n+b, b^n+a) = 1$ for all $n$, and the sequence is eventually constant.

If $d = 2$, then $a$ and $b$ must be both odd. Let $a = 2a'$ and $b = 2b'$, where $a'$ and $b'$ are positive integers. Then the sequence becomes
\[
\gcd((2a')^n + 2b', (2b')^n + 2a') = \gcd(2(a'^n+b'), 2(b'^n+a')) = 2\gcd(a'^n+b', b'^n+a').
\]

Since $\gcd(a',b') = 1$, we have $\gcd(a'^n+b', b'^n+a') = 1$ for all $n$. Therefore, the sequence is eventually constant in this case as well.

In conclusion, the sequence is eventually constant for all pairs of positive integers $(a,b)$ such that $\gcd(a,b) = 1$.
