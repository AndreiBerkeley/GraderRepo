Let \(O\) be the intersection point of lines \(AD\) and \(BE\). Since \(BC=DE\), we have \(BO=OE\). Also, since \(TB=TD\) and \(\angle ABT = \angle TEA\), we have \(\triangle TBO \cong \triangle TEO\) by SAS congruence. Thus, \(\angle TBO = \angle TOE\).

Now, since \(AB \parallel RS\), we have \(\angle ABR = \angle RSE\). Also, since \(AB \parallel RS\) and \(AE \parallel PQ\), we have \(\angle ABR = \angle RSE = \angle QCT\). Therefore, \(\angle RSE = \angle QCT\).

Since \(\angle TBO = \angle TOE\), we have \(\angle TBP = \angle TEP\). Also, since \(\angle TBP = \angle TEP\) and \(\angle TBC = \angle TED\), we have \(\angle PBC = \angle EDC\). Therefore, \(\angle PBC = \angle EDC\).

Now, we have
\[\angle PSR = \angle RSE + \angle PBC = \angle QCT + \angle EDC = \angle QCD.\]
Also, we have
\[\angle PSQ = \angle QCT + \angle TCB = \angle QCT + \angle TBC = \angle QCD.\]
Therefore, \(\angle PSR = \angle PSQ\), which implies that the points \(P\), \(S\), \(Q\), \(R\) lie on a circle.
