Let's denote the length of the interval \([a,b]\) as \(|a-b|\). We are given that the sum of the lengths of the intervals \([x_{2i-1}^{2k-1}, x_{2i}^{2k-1}]\) is equal to 1 for all integers \(k\) with \(1 \leq k \leq m\).

The length of the interval \([x_{2i-1}^{2k-1}, x_{2i}^{2k-1}]\) is \(|x_{2i}^{2k-1} - x_{2i-1}^{2k-1}| = |x_{2i} - x_{2i-1}|^{2k-1}\).

Given that the sum of the lengths of the intervals is 1 for all \(k\) with \(1 \leq k \leq m\), we have
\[
\sum_{i=1}^{n} |x_{2i} - x_{2i-1}|^{2k-1} = 1
\]
for all \(1 \leq k \leq m\).

Now, let's consider the case where \(k = 1\). Since the sum of the lengths of the intervals is 1, we have
\[
\sum_{i=1}^{n} |x_{2i} - x_{2i-1}| = 1.
\]

This implies that the sum of the absolute differences of consecutive \(x_i\) is 1. This is possible if and only if the \(x_i\) are equally spaced. Thus, we have
\[
x_{2i} - x_{2i-1} = \frac{1}{n}
\]
for all \(i = 1, 2, \ldots, n\).

Now, let's consider the case where \(k = 2\). We have
\[
\sum_{i=1}^{n} |x_{2i} - x_{2i-1}|^3 = 1.
\]

Substitute \(x_{2i} - x_{2i-1} = \frac{1}{n}\) into the equation above, we get
\[
\sum_{i=1}^{n} \left(\frac{1}{n}\right)^3 = \frac{1}{n^2} = 1.
\]

This implies that \(n = 1\), which is not possible since \(n\) is a positive integer. Therefore, the largest integer \(m\) is \(m = 1\).
