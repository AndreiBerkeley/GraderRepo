Let $a_1, a_2, \ldots, a_n$ be the numbers in a row of the table. Since the numbers in each row can be permuted to form an arithmetic progression, we have $a_i = a_1 + (i-1)d$ for some common difference $d$. Thus, the sum of the numbers in the row is given by
\begin{align*}
S &= a_1 + a_2 + \cdots + a_n \\
&= na_1 + d\left(1 + 2 + \cdots + n\right) \\
&= na_1 + \frac{n(n+1)}{2}d.
\end{align*}

Similarly, let $b_1, b_2, \ldots, b_n$ be the numbers in a column of the table. Since the numbers in each column can be permuted to form an arithmetic progression, we have $b_i = b_1 + (i-1)e$ for some common difference $e$. Thus, the sum of the numbers in the column is given by
\begin{align*}
T &= b_1 + b_2 + \cdots + b_n \\
&= nb_1 + e\left(1 + 2 + \cdots + n\right) \\
&= nb_1 + \frac{n(n+1)}{2}e.
\end{align*}

Since the sum of the numbers in each row is equal to the sum of the numbers in each column, we have $na_1 + \frac{n(n+1)}{2}d = nb_1 + \frac{n(n+1)}{2}e$. This simplifies to $a_1 + \frac{n+1}{2}d = b_1 + \frac{n+1}{2}e$.

Therefore, in order to transform any row-valid arrangement into a column-valid arrangement by permuting the numbers in each row, we must have $a_1 + \frac{n+1}{2}d = b_1 + \frac{n+1}{2}e$ for all rows. This implies that the common differences $d$ and $e$ must be the same for all rows.

Since $n \geq 3$, we can choose $n = 3$ as a counterexample. Consider the row-valid arrangement where the rows are $(1, 2, 3)$, $(4, 5, 6)$, and $(7, 8, 9)$. This arrangement cannot be transformed into a column-valid arrangement by permuting the numbers in each row, as the common differences in the rows are different.

Therefore, it is not possible to transform any row-valid arrangement into a column-valid arrangement by permuting the numbers in each row for any value of $n$.
