Solution 1.  Cartesian coordinates  

Put the whole configuration in a rectangular system of co-ordinates in the
following convenient way

•  let $C$ be the origin $(0,0)$;  
•  let the $x$–axis be the line $BC$;  
•  fix $B$ at the point $(b,0)$ with $b\ne 0$;  
•  write $A=(p,q)$ with $q>0$ (the triangle is acute).

Throughout the proof the only assumption that will be used is $q\neq 0$; the
sign of $b$ is irrelevant.

--------------------------------------------------------------------
1.  The auxiliary points in co-ordinates
--------------------------------------------------------------------

(i)  The foot $F$ of the perpendicular from $C$ to $AB$.

The direction vector of the line $AB$ is
$\mathbf v=(p-b,q)$ and therefore
$AB$ has equation
$\displaystyle (p-b)\,y-q\,(x-b)=0$.
Because $F$ is the orthogonal projection of the origin on $AB$ we must have
$\mathbf v\cdot\overrightarrow{CF}=0$,  that is
\[
(p-b)\,x_{F}+q\,y_{F}=0\qquad\text{and}\qquad
\overrightarrow{BF}=t\mathbf v\; (t\in\mathbb R).
\]
Solving these two relations one obtains
\[
t=-\,\frac{b\,(p-b)}{(p-b)^2+q^2},\qquad
F=\Bigl(b+t(p-b),\;tq\Bigr)
      =\Bigl(b-\frac{b(p-b)^2}{(p-b)^2+q^2},\;
             -\frac{b(p-b)q}{(p-b)^2+q^2}\Bigr).      \tag{1}
\]

(ii)  The orthocentre $H$ and the reflection $P$ of $H$ in $BC$.

Because $BC$ is the $x$–axis, the altitude from $A$ is the vertical line
$x=p$.  
The altitude from $B$ is orthogonal to $AC$, hence it has slope
$-\dfrac{p}{q}$ and equation $y=-\dfrac{p}{q}(x-b)$.  
Intersecting these two lines gives
\[
H=\Bigl(p,\,-\,\frac{p(p-b)}{q}\Bigr)\quad\Longrightarrow\quad
P=\Bigl(p,\;\frac{p(p-b)}{q}\Bigr).                           \tag{2}
\]

--------------------------------------------------------------------
2.  The circle $(AFP)$
--------------------------------------------------------------------

Let its equation be written in the usual form
\[
x^{2}+y^{2}+Ax+By+C=0.                                    \tag{$\ast$}
\]
Because the three points $A,F,P$ lie on this circle, substituting their
co-ordinates gives three linear relations for the unknowns $A,B,C$:

\[
\left\{
\begin{aligned}
&p^{2}+q^{2}+Ap+Bq+C=0,\\[2pt]
&x_{F}^{2}+y_{F}^{2}+Ax_{F}+By_{F}+C=0,\\[2pt]
&p^{2}+\dfrac{p^{2}(p-b)^{2}}{q^{2}}+Ap+
     B\cdot\dfrac{p(p-b)}{q}+C=0.
\end{aligned}
\right.
\]
Subtracting the first equation from the third eliminates $A$ and $C$ and
yields
\[
\Bigl(\dfrac{p(p-b)}{q}-q\Bigr)B
          =q^{2}-\dfrac{p^{2}(p-b)^{2}}{q^{2}}
          =\dfrac{q^{4}-p^{2}(p-b)^{2}}{q^{2}}.
\]
Hence
\[
B=-\,(q+\dfrac{p(p-b)}{q}).                              \tag{3}
\]

Now eliminate $B$ and $C$ by subtracting the first equation
from the second one.  Using (1) for $F$ and (3) for $B$,
after a routine but lengthy simplification one obtains

\[
A=0.                                                      \tag{4}
\]

So in the equation ($\ast$) of the circumcircle of $\triangle AFP$ the
coefficient of $x$ is \emph{identically equal to $0$}.

--------------------------------------------------------------------
3.  Intersections with the $x$-axis
--------------------------------------------------------------------

Putting $y=0$ in ($\ast$) gives
\[
x^{2}+C=0 .
\]
Because $A=0$ this quadratic has its two real roots symmetrical with respect
to the origin:
if the roots are $x_{1},x_{2}$, then $x_{1}+x_{2}=0$ and $x_{1}x_{2}=C$.

In geometric language this means that the two intersection points
$X,Y$ of the circle $(AFP)$ with the $x$-axis (= the line $BC$)
satisfy
\[
CX = |x_{1}| = |x_{2}| = CY .
\]

--------------------------------------------------------------------
4.  Translation back to the original wording
--------------------------------------------------------------------

The $x$–axis is the original side $BC$ and the origin is the vertex $C$;
therefore the equalities $|x_{1}|=|x_{2}|$ obtained above are precisely

\[
\boxed{\,CX=CY\,}.
\]

Thus the required statement has been proved.

--------------------------------------------------------------------
Remark
--------------------------------------------------------------------
The whole computation hinges on one single fact: the coefficient $A$
of $x$ in the circle–equation vanishes.  Once $A=0$ is known,
symmetry of the two intersection abscissæ is immediate.
Every step is purely algebraic and completely elementary,
so the proof is rigorous and valid for \emph{every} acute triangle
$ABC$.