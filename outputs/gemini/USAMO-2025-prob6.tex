Answering the question: Find all positive integers $n$ for which there exists a permutation $\sigma$ of the set $\{1, 2, \dots, n\}$ such that the set $\{i + \sigma(i) \mid i \in \{1, 2, \dots, n\}\}$ contains exactly two distinct values.

Let $S = \{i + \sigma(i) \mid i \in \{1, 2, \dots, n\}\}$. The problem states that $|S|=2$.
Let the two distinct values in $S$ be $a$ and $b$. Without loss of generality, let's assume $a < b$.

**1. Initial Analysis and Setup**

Every number $i \in \{1, 2, \dots, n\}$ must satisfy either $i + \sigma(i) = a$ or $i + \sigma(i) = b$.
This induces a partition of the set $\{1, 2, \dots, n\}$ into two non-empty subsets:
$A = \{i \in \{1, 2, \dots, n\} \mid i + \sigma(i) = a\}$
$B = \{i \in \{1, 2, \dots, n\} \mid i + \sigma(i) = b\}$
We know $A \cup B = \{1, 2, \dots, n\}$, $A \cap B = \emptyset$, and $A, B$ are non-empty (otherwise $|S|=1$).

For any $i \in A$, we have $\sigma(i) = a-i$.
For any $i \in B$, we have $\sigma(i) = b-i$.

Let's sum the values of $i + \sigma(i)$ over all $i$:
$$ \sum_{i=1}^n (i + \sigma(i)) = \sum_{i=1}^n i + \sum_{i=1}^n \sigma(i) $$
Since $\sigma$ is a permutation of $\{1, 2, \dots, n\}$, the set of values $\{\sigma(1), \dots, \sigma(n)\}$ is the same as $\{1, \dots, n\}$. Thus, $\sum_{i=1}^n \sigma(i) = \sum_{i=1}^n i = \frac{n(n+1)}{2}$.
The total sum is $2 \cdot \frac{n(n+1)}{2} = n(n+1)$.

Let $k = |A|$, so $|B|=n-k$. We know $1 \le k \le n-1$.
The sum can also be calculated as:
$$ \sum_{i=1}^n (i + \sigma(i)) = \sum_{i \in A} a + \sum_{i \in B} b = ka + (n-k)b $$
This gives the relation:
$$ ka + (n-k)b = n(n+1) $$

**2. The Structure of the Permutation**

The crucial insight is to understand how $\sigma$ acts on the sets $A$ and $B$.
Let's analyze the structure of $\sigma$. An element from $A$ can be mapped into $A$ or $B$. Similarly for an element from $B$. Let's define the following sets:
$A_A = \{i \in A \mid \sigma(i) \in A\}$, $A_B = \{i \in A \mid \sigma(i) \in B\}$
$B_A = \{i \in B \mid \sigma(i) \in A\}$, $B_B = \{i \in B \mid \sigma(i) \in B\}$

Let's consider an element $i \in A_B$. By definition, $i \in A$ and $\sigma(i) \in B$.
So, $\sigma(i) = a-i$.
Let's apply $\sigma$ again. Since $\sigma(i) \in B$, we have $\sigma(\sigma(i)) = b - \sigma(i)$.
$\sigma^2(i) = b - (a-i) = i + (b-a)$.
Let $d = b-a$. Since $a<b$, $d$ is a positive integer.
So, for any $i \in A_B$, we have $\sigma^2(i) = i+d$.

Since $\sigma$ is a permutation on the finite set $\{1, \dots, n\}$, $\sigma^2$ is also a permutation on this set. Thus, $\sigma^2$ must map the subset $A_B$ to itself.
If $A_B$ is non-empty, let $i_{max}$ be the largest element in $A_B$.
Then $\sigma^2(i_{max}) = i_{max} + d$ must also be an element of $A_B$.
However, $d>0$, so $i_{max} + d > i_{max}$. This contradicts the fact that $i_{max}$ is the largest element of $A_B$.
Therefore, the set $A_B$ must be empty.

A similar argument holds for $B_A$. For any $j \in B_A$, we have $j \in B$ and $\sigma(j) \in A$.
$\sigma(j) = b-j$.
$\sigma^2(j) = a - \sigma(j) = a - (b-j) = j - (b-a) = j-d$.
If $B_A$ is non-empty, let $j_{min}$ be the smallest element in $B_A$.
Then $\sigma^2(j_{min}) = j_{min} - d$ must also be in $B_A$.
But $d>0$, so $j_{min}-d < j_{min}$, which contradicts the choice of $j_{min}$.
Therefore, the set $B_A$ must also be empty.

The fact that $A_B=\emptyset$ and $B_A=\emptyset$ means that $\sigma$ maps no element of $A$ to $B$ and no element of $B$ to $A$.
This implies that $\sigma$ maps $A$ to $A$ and $B$ to $B$.
So, $\sigma$ is composed of two separate permutations: $\sigma_A$ on $A$ and $\sigma_B$ on $B$.
- For $i \in A$, $\sigma(i) = a-i$ must be an element of $A$. So, $\{a-i \mid i \in A\} = A$.
- For $j \in B$, $\sigma(j) = b-j$ must be an element of $B$. So, $\{b-j \mid j \in B\} = B$.

**3. The Symmetry of the Sets A and B**

The condition $\{a-i \mid i \in A\} = A$ implies that the set $A$ must be symmetric about the value $a/2$.
To see this, sum over all elements in $A$:
$\sum_{i \in A} (a-i) = \sum_{j \in A} j$
$|A|a - \sum_{i \in A} i = \sum_{i \in A} i$
$ka = 2 \sum_{i \in A} i \implies a = 2 \frac{\sum_{i \in A} i}{k} = 2 \cdot \text{avg}(A)$.

Similarly for $B$:
$(n-k)b = 2 \sum_{j \in B} j \implies b = 2 \frac{\sum_{j \in B} j}{n-k} = 2 \cdot \text{avg}(B)$.

For $a$ and $b$ to be distinct, we need $\text{avg}(A) \ne \text{avg}(B)$.

**4. Existence of such permutations**

The problem is now reduced to finding all $n$ for which we can partition $\{1, \dots, n\}$ into two non-empty sets $A$ and $B$ such that:
1. $A$ is symmetric about some number $c_A = a/2$.
2. $B$ is symmetric about some number $c_B = b/2$.
3. $c_A \ne c_B$ (which implies $a \ne b$).

Let's check small values of $n$.
- For $n=1$, the set is $\{1\}$. It cannot be partitioned into two non-empty subsets. Also, if we take $A=\{1\}, B=\emptyset$, then $|S|=1$. So $n=1$ is not a solution.

- For any $n \ge 2$, we can construct such a partition.
Let's choose a value $k \in \{1, 2, \dots, n-1\}$. Consider the partition:
$A = \{1, 2, \dots, k\}$
$B = \{k+1, k+2, \dots, n\}$

Let's check the symmetry properties:
For set $A = \{1, \dots, k\}$, the average is $\frac{k+1}{2}$. So we set $c_A = \frac{k+1}{2}$, which makes $a = k+1$.
The set of images is $\{a-i \mid i \in A\} = \{k+1-i \mid i \in \{1,\dots,k\}\} = \{k, k-1, \dots, 1\} = A$.
This is a valid choice. The values $a=k+1$ and $\sigma_A(i)=k+1-i$ for $i \in A$ work.

For set $B = \{k+1, \dots, n\}$, the average is $\frac{(k+1)+n}{2}$. We set $c_B = \frac{n+k+1}{2}$, which makes $b=n+k+1$.
The set of images is $\{b-j \mid j \in B\} = \{n+k+1-j \mid j \in \{k+1,\dots,n\}\} = \{n, n-1, \dots, k+1\} = B$.
This is also a valid choice. The values $b=n+k+1$ and $\sigma_B(j)=n+k+1-j$ for $j \in B$ work.

Since $n \ge 2$, $n+k+1 > k+1$, so $a \ne b$.
We can define the permutation $\sigma$ on $\{1, 2, \dots, n\}$ as:
$$ \sigma(i) = \begin{cases} k+1-i & \text{if } 1 \le i \le k \\ n+k+1-i & \text{if } k+1 \le i \le n \end{cases} $$
Let's verify this is a permutation. $\sigma$ maps $A$ to $A$ and $B$ to $B$. Since $A$ and $B$ are disjoint, $\sigma$ is injective on $A \cup B$ and thus is a permutation of $\{1, \dots, n\}$.

The set of sums $\{i+\sigma(i)\}$ is:
- For $i \in A$, $i+\sigma(i) = i+(k+1-i)=k+1=a$.
- For $i \in B$, $i+\sigma(i) = i+(n+k+1-i)=n+k+1=b$.

So the set of sums is $\{k+1, n+k+1\}$. Since $n \ge 2$, these are two distinct values.
This construction works for any integer $n \ge 2$ (by choosing any valid $k$, for instance $k=1$).

Example for $n=4$:
Let $k=2$. Then $A=\{1,2\}, B=\{3,4\}$.
$a=2+1=3$. $\sigma(1)=2, \sigma(2)=1$.
$b=4+2+1=7$. $\sigma(3)=4, \sigma(4)=3$.
$\sigma$ is the permutation $(1\;2)(3\;4)$.
$1+\sigma(1)=3, 2+\sigma(2)=3$.
$3+\sigma(3)=7, 4+\sigma(4)=7$.
The set of sums is $\{3,7\}$, which has size 2.

**Conclusion**

A permutation $\sigma$ with the desired property exists if and only if $n \ge 2$. For $n=1$, it's impossible to have two distinct sum values. For any $n \ge 2$, we have demonstrated a construction that satisfies the condition.

Final Answer: The set of all positive integers $n$ is $\{n \in \mathbb{Z}^+ \mid n \ge 2\}$.