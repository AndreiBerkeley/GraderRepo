Let $I$ be the point of concurrency of the angle bisectors of $\angle A$, $\angle C$, and $\angle E$. Denote the intersection of $AI$ with $BC$, $CD$, $DE$, and $EF$ as $P$, $Q$, $R$, and $S$, respectively. 

Since $\angle A = \angle C = \angle E$, we have $\angle B = \angle D = \angle F$. Let $J$ be the incenter of $\triangle BCD$. By the Angle Bisector Theorem, we have
\[\frac{BP}{PC} = \frac{BI}{CI} = \frac{BJ}{CJ}.\]
Similarly, we have
\[\frac{CQ}{QD} = \frac{CK}{DK} = \frac{CL}{DL},\]
and
\[\frac{DR}{RE} = \frac{DL}{EL} = \frac{DM}{EM}.\]
Multiplying these three equations together, we get
\[\frac{BP}{PC} \cdot \frac{CQ}{QD} \cdot \frac{DR}{RE} = \frac{BJ}{CJ} \cdot \frac{CK}{DK} \cdot \frac{DL}{EL} = \frac{BJ}{CJ} \cdot \frac{CK}{DK} \cdot \frac{DM}{EM} = 1.\]
By the Converse of the Angle Bisector Theorem, we conclude that $IJ$ is the angle bisector of $\angle BID$. Similarly, we can show that $IJ$ is the angle bisector of $\angle DID$ and $\angle FID$. Therefore, $IJ$ is the angle bisector of $\angle BID$, $\angle DID$, and $\angle FID$, which implies that the angle bisectors of $\angle B$, $\angle D$, and $\angle F$ are concurrent.
