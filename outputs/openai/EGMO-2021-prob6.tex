Let  

S(m)=∑_{k=1}^{m}⌊m/k⌋  (m=1,2,3, …).

The required equation is  

S(m)=n²+a   (m,n∈ℕ, a∈ℕ₀). (1)

We show that for a suitable a the set of solutions (m,n) is larger than one
million.

────────────────────────────────────────────────────────
1.  A convenient upper bound for S(m)
────────────────────────────────────────────────────────
Because ⌊m/k⌋≤m/k,

 S(m)=∑_{k=1}^{m}⌊m/k⌋≤m∑_{k=1}^{m}1/k = m H_m ,  

where H_m is the m-th harmonic number.  Since H_m<ln m+1 for m≥1,  

 S(m)<m(ln m+1)<2m ln m  (m≥3). (2)

────────────────────────────────────────────────────────
2.  “Defects from the nearest square’’
────────────────────────────────────────────────────────
For every m put  

 b(m)=S(m)−⌊√{S(m)}⌋². (3)

Obviously b(m) is a non–negative integer and

 0≤b(m)<2√{S(m)} ≤2√{2m ln m}  (by (2)). (4)

────────────────────────────────────────────────────────
3.  How many different values can b(m) take?
────────────────────────────────────────────────────────
Fix a large integer M and look at the first M numbers b(1),…,b(M).

By (4) each of them lies in  

 [0,2√{2M ln M}).  

Hence the set {b(1),…,b(M)} has cardinality at most  

 L:=⌈2√{2M ln M}⌉. (5)

────────────────────────────────────────────────────────
4.  Pigeonhole principle
────────────────────────────────────────────────────────
Distributing M integers among at most L boxes, some value a occurs at least  

 Q:=⌈M/L⌉≥ M/(2√{2M ln M}+1) > √M /(3√{2 ln M}). (6)

────────────────────────────────────────────────────────
5.  Choosing M
────────────────────────────────────────────────────────
We need Q>1,000,000.  
Take M:=10¹⁵.  Then √M=10^{7.5}=3.16·10⁷ and ln M≈34.54, whence

 Q> 3.16·10⁷ /(3√{2·34.54}) ≈ 1.27·10⁶ > 10⁶.

Thus for this M there is a non-negative integer a that occurs as b(m) for
more than one million different m (namely, for at least Q of the integers
1,…,M).

────────────────────────────────────────────────────────
6.  Constructing the solutions
────────────────────────────────────────────────────────
For every m with b(m)=a we have, by (3),

 S(m)=⌊√{S(m)}⌋²+a=n²+a  with n:=√{S(m)−a}∈ℕ.

Hence each of those m furnishes a solution (m,n) of (1).
Because they are distinct m’s, the pairs (m,n) are distinct as well; their
number exceeds 1,000,000.

────────────────────────────────────────────────────────
7.  Conclusion
────────────────────────────────────────────────────────
Yes.  There exists a non-negative integer a (e.g. the most frequent value
of b(m) among 1≤m≤10¹⁵) for which the equation

 ⌊m/1⌋+⌊m/2⌋+⋯+⌊m/m⌋ = n²+a

has more than one million different solutions in positive integers (m,n).