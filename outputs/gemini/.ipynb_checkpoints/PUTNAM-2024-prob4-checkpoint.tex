Of course. Here is a complete and rigorous, step-by-step solution.

### Problem Statement

Find all pairs of positive integers $(m, n)$ that satisfy the equation:
$$m^2 - 2^n = 1$$

### Solution

We are looking for positive integer solutions $(m, n)$ to the equation $m^2 - 2^n = 1$.

**Step 1: Rearrange the equation and factor**

The given equation can be rearranged to isolate the power of 2:
$$m^2 - 1 = 2^n$$
The left-hand side is a difference of squares, which can be factored as $(m-1)(m+1)$.
So, our equation becomes:
$$(m-1)(m+1) = 2^n$$

**Step 2: Analyze the factors**

The right-hand side of the equation, $2^n$, is a power of 2. This means that its only prime factor is 2. Consequently, the factors on the left-hand side, $(m-1)$ and $(m+1)$, must also be powers of 2.

Let's formalize this. If either $(m-1)$ or $(m+1)$ had an odd prime factor $p$, then their product would also have the odd prime factor $p$. However, their product is $2^n$, which has no odd prime factors. Therefore, both $(m-1)$ and $(m+1)$ must be powers of 2.

We can write:
$$m-1 = 2^a$$
$$m+1 = 2^b$$
for some non-negative integers $a$ and $b$.

**Step 3: Establish relationships between the exponents**

Since $m$ is a positive integer, $m \ge 1$.
This implies $m-1 \ge 0$ and $m+1 \ge 2$.
From $m-1 = 2^a$, we know $a$ must be a non-negative integer.
From $m+1 = 2^b$, we know $b$ must be a positive integer, so $b \ge 1$.

Also, it is clear that $m+1 > m-1$. Therefore, $2^b > 2^a$, which implies $b > a$.

Substituting these into the factored equation:
$$(2^a)(2^b) = 2^n$$
$$2^{a+b} = 2^n$$
This gives us the relationship $n = a+b$.

**Step 4: Solve for the exponents a and b**

We have a system of two equations involving $m, a, b$:
1. $m-1 = 2^a$
2. $m+1 = 2^b$

Let's subtract the first equation from the second to eliminate $m$:
$$(m+1) - (m-1) = 2^b - 2^a$$
$$2 = 2^b - 2^a$$

Since we established that $b > a$, we can factor out the smaller power of 2, which is $2^a$:
$$2 = 2^a(2^{b-a} - 1)$$

We are looking for integer solutions for $a$ and $b$. The left side is $2^1$. Let's analyze the factors on the right side:
*   $2^a$ is a power of 2.
*   Since $b>a$, the exponent $b-a$ is a positive integer, so $b-a \ge 1$.
    *   If $b-a \ge 1$, then $2^{b-a}$ is an even integer.
    *   Therefore, $2^{b-a} - 1$ is an odd integer.

So, the equation $2^1 = 2^a(2^{b-a} - 1)$ expresses $2$ as a product of a power of two ($2^a$) and an odd number ($2^{b-a} - 1$). The only way this is possible is if the power of two on the right matches the power of two on the left, and the odd factor is 1.

This gives us two conditions:
1. $2^a = 2^1 \implies a=1$
2. $2^{b-a} - 1 = 1$

Now we solve the second condition using the value of $a=1$:
$$2^{b-1} - 1 = 1$$
$$2^{b-1} = 2$$
$$2^{b-1} = 2^1$$
The exponents must be equal, so:
$$b-1 = 1 \implies b=2$$

We have found a unique pair of exponents $(a,b) = (1,2)$. These values are consistent with our conditions $a \ge 0$ and $b > a$.

**Step 5: Determine m and n**

Now we can find the values of $m$ and $n$.
Using $m-1 = 2^a$:
$$m-1 = 2^1 = 2$$
$$m = 3$$

We can verify this with $m+1 = 2^b$:
$$m+1 = 2^2 = 4$$
$$m = 3$$
The value $m=3$ is consistent.

Now we find $n$ using the relation $n=a+b$:
$$n = 1+2 = 3$$

Thus, we have found a single potential solution pair $(m, n) = (3, 3)$.

**Step 6: Verification**

Let's check if the pair $(m,n)=(3,3)$ satisfies the original equation $m^2 - 2^n = 1$:
$$3^2 - 2^3 = 9 - 8 = 1$$
The equation holds true. Since all steps in our derivation followed from logical necessity, this is the unique solution.

### Conclusion

The only pair of positive integers $(m, n)$ that satisfies the equation $m^2 - 2^n = 1$ is $(3, 3)$.

---

### Alternative approach using Catalan's Conjecture

The problem is a specific case of Catalan's Conjecture (now Mihailescu's Theorem). The theorem states that the only solution in natural numbers of the equation $x^a - y^b = 1$ for $x, y, a, b > 1$ is $x=3, a=2, y=2, b=3$.

Our equation is $m^2 - 2^n = 1$.
This fits the form $x^a - y^b = 1$ with $x=m, a=2, y=2, b=n$.

We must consider the constraints on the exponents. In our problem, $m, n$ are positive integers. The exponent on $m$ is $2$, which is greater than 1.

*   **Case 1: $n=1$**
    The equation becomes $m^2 - 2^1 = 1$, which gives $m^2 = 3$. This has no integer solution for $m$.

*   **Case 2: $n>1$**
    In this case, all exponents are greater than 1 ($a=2>1$ and $b=n>1$). According to Mihailescu's Theorem, the only solution is $x=m=3, a=2, y=2, b=n=3$.
    This gives the solution $(m,n) = (3,3)$.

This confirms our result from the elementary method. The step-by-step factorization method is self-contained and does not require this powerful theorem, but it's a nice way to verify the uniqueness of the solution.