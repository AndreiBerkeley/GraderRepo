Let \( H \) be the orthocenter of triangle \( ABC \), and let \( D \) be the foot of the altitude from \( A \) to \( BC \).  
Since \( E \) and \( F \) are the intersections of the external bisector of angle \( A \) with the altitudes of \( ABC \) through \( B \) and \( C \) respectively, we have \( \angle EHA = \angle FHA = 90^\circ \).  
This implies that \( E \), \( F \), \( H \) are collinear.  

Since \( \angle EMA = \angle BCA \) and \( \angle ANF = \angle ABC \), we have \( \angle EMB = \angle EMA = \angle BCA \) and \( \angle ANB = \angle ANF = \angle ABC \).  
This implies that \( \triangle EMB \sim \triangle BCA \) and \( \triangle ANB \sim \triangle ABC \).  
Hence, we have \( \angle EBM = \angle ACB \) and \( \angle ANB = \angle ABC \).  

Now, consider the cyclic quadrilateral \( AENF \).  
We have \( \angle AEN = \angle ABC \) and \( \angle ANE = \angle ACB \).  
Since \( \angle ANB = \angle ABC \), we have \( \angle ANE = \angle ANB \), which implies that \( \angle NEB = \angle NAB \).  
Therefore, \( \triangle NEB \sim \triangle NAB \), which implies that \( \angle NEB = \angle NAB = \angle ACB \).  

Hence, we have \( \angle EBM = \angle NEB \), which implies that \( E \), \( F \), \( N \), \( M \) are concyclic.  
Therefore, the points \( E \), \( F \), \( N \), \( M \) lie on a circle.
