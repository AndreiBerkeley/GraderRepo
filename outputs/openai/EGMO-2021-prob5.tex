Let  

n = |P| = 2021.  
Label the points clockwise by their directions from O:  

P₁ , P₂ , … , Pₙ   (angles 0° < θ₁ < θ₂ < … < θₙ < 360°).

--------------------------------------------------------------------
1.  When does a triangle contain the origin?
--------------------------------------------------------------------
For three points P_i , P_j , P_k (in circular order i<j<k) put on a
large circle centred at O, the origin lies inside △P_iP_jP_k
iff the three points are **not** contained in any open half–plane whose
border passes through O.  
Equivalently:

“△P_iP_jP_k is fat  ⇔  the clockwise angular distance from the
first to the last of the three points is > 180°.”

--------------------------------------------------------------------
2.  Counting the non-fat (bad) triangles
--------------------------------------------------------------------
Fix a point P_i .
Let f(i) be the number of other points that lie strictly within
the 180° clockwise sector starting at P_i.
(The sector endpoint is excluded because no two points are on one ray.)

If two further points of that sector are chosen, the three vertices lie
in that same 180° half-plane, hence the triangle is **not** fat.
Conversely every non-fat triangle has a unique first vertex whose two
partners lie in its clockwise 180° sector.  Thus

bad triangles  =  Σ_{i=1}^{n} C(f(i),2).      (1)

--------------------------------------------------------------------
3.  The multiset {f(i)}
--------------------------------------------------------------------
For every unordered pair {P_a , P_b} exactly one of the two
clockwise arcs P_a→P_b , P_b→P_a is < 180°; hence that pair is counted
in f(a) **or** in f(b), but not in both.
Therefore

Σ_{i=1}^{n} f(i) = C(n,2) = n(n−1)/2.          (2)

--------------------------------------------------------------------
4.  Minimising Σ C(f(i),2)
--------------------------------------------------------------------
The function ϕ(k)=C(k,2)=k(k−1)/2 is convex for k∈ℤ.
With a fixed sum (2), Jensen/Karamata tells us that
Σϕ(f(i)) is minimised when the f(i)’s are as equal as possible.

Because n is odd,

average f(i) = ( n(n−1)/2 ) ÷ n = (n−1)/2 = 1010.

The sum (2) is an exact multiple of n, so we can have  
f(1)=f(2)=…=f(n)=1010, giving the minimum

Σ C(f(i),2) = n·C(1010,2).                   (3)

--------------------------------------------------------------------
5.  Realising the minimum
--------------------------------------------------------------------
Place the points as the vertices of a regular 2021-gon centred at O.
Then every 180° sector starting at a vertex contains
exactly (n−1)/2 = 1010 other vertices, i.e. f(i)=1010 for every i.
The two given conditions (‘no three collinear’, ‘no two on a line
through O’) are satisfied, so (3) is attainable.

Hence (3) really is the least possible number of non-fat triangles,
so the number of fat triangles is maximised for this configuration.

--------------------------------------------------------------------
6.  The maximum number of fat triangles
--------------------------------------------------------------------
Total triangles:        C(n,3) = C(2021,3).

Non-fat triangles:      n·C(1010,2).

Fat triangles:

C(2021,3) – 2021·C(1010,2)
 = (2021·2020·2019)/6  –  2021·(1010·1009)/2
 = 1 373 734 330       –   1 029 790 445
 = 343 943 885.

--------------------------------------------------------------------
Answer
--------------------------------------------------------------------
The maximum possible number of fat triangles is  

343 943 885.