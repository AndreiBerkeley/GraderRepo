Let \( n \) be the number of points in \( P \). We are given that \( n = 2021 \). Let \( f(n) \) be the maximum number of fat triangles that can be formed with \( n \) points. We will prove that \( f(n) = \binom{n}{3} - \binom{n-1}{3} \).

Consider a set of \( n \) points in the plane such that no three points are collinear and no two points lie on a line through the origin. Let \( A \) be the set of all triangles formed by choosing \( 3 \) points from the set \( P \). We will show that the maximum number of fat triangles is achieved when the origin is inside exactly one of the triangles in \( A \).

Let \( T \) be a fat triangle with vertices in \( P \). If the origin is inside \( T \), then \( T \) is a fat triangle. If the origin is outside \( T \), then we can draw a line through the origin that intersects \( T \) at exactly one point. This divides \( T \) into two triangles, one of which is fat. Thus, each non-fat triangle can be split into two fat triangles.

Now, let's count the number of fat triangles. If the origin is inside exactly one of the triangles in \( A \), then there are \( \binom{n}{3} - \binom{n-1}{3} \) fat triangles. This is because we choose \( 3 \) points from \( P \) to form a triangle, and then choose one of the \( n \) triangles to contain the origin.

Therefore, the maximum number of fat triangles is \( f(n) = \binom{n}{3} - \binom{n-1}{3} = \binom{2021}{3} - \binom{2020}{3} = 1365702660 \).
