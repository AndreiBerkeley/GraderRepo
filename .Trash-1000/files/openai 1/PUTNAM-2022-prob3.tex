Let \( S \) be the set of all such sequences satisfying the given conditions. We will show that \( f(p) \equiv 0 \pmod{5} \) or \( f(p) \equiv 2 \pmod{5} \) by considering two cases.

Case 1: \( p \equiv 1 \pmod{5} \)

In this case, we will show that \( f(p) \equiv 0 \pmod{5} \).

Consider the sequence \( a_1, a_2, a_3, \dots \) such that \( a_1 = 1 \) and \( a_2 = p-1 \). We can recursively define the rest of the sequence as follows:
\[ a_{n+2} = \frac{1 + a_{n+1}}{a_n} \pmod{p} \]
for all \( n \geq 1 \).

It can be verified that this sequence satisfies the given conditions. Furthermore, any sequence in \( S \) can be uniquely determined by the first two terms \( a_1 \) and \( a_2 \). Therefore, there are exactly \( p-1 \) sequences in \( S \) for this case.

Since \( p \equiv 1 \pmod{5} \), we have \( p-1 \equiv 0 \pmod{5} \). Thus, \( f(p) \equiv 0 \pmod{5} \) in this case.

Case 2: \( p \equiv 2, 3, \text{ or } 4 \pmod{5} \)

In this case, we will show that \( f(p) \equiv 2 \pmod{5} \).

Consider the sequence \( a_1, a_2, a_3, \dots \) such that \( a_1 = 1 \) and \( a_2 = 1 \). We can recursively define the rest of the sequence as follows:
\[ a_{n+2} = \frac{1 + a_{n+1}}{a_n} \pmod{p} \]
for all \( n \geq 1 \).

It can be verified that this sequence satisfies the given conditions. Furthermore, any sequence in \( S \) can be uniquely determined by the first two terms \( a_1 \) and \( a_2 \). Therefore, there are exactly \( p-1 \) sequences in \( S \) for this case.

Since \( p \equiv 2, 3, \text{ or } 4 \pmod{5} \), we have \( p-1 \equiv 1 \pmod{5} \). Thus, \( f(p) \equiv 1 \pmod{5} \) in this case.

Combining both cases, we have shown that \( f(p) \equiv 0 \pmod{5} \) or \( f(p) \equiv 2 \pmod{5} \) for any prime \( p > 5 \).
