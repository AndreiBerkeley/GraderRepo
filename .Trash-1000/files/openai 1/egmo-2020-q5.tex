Let $O$ be the center of the circumcircle $\Gamma$ of triangle $ABC$. Since $PB = PC$, we have $\angle PBC = \angle PCB = \alpha$.

Since $PA$ is equal to the radius of $\Gamma$, we have $PA = PO = R$, where $R$ is the radius of $\Gamma$. Let $M$ be the midpoint of $PB$. Since $M$ lies on the perpendicular bisector of $PB$, we have $PM = BM = \frac{1}{2}PB$. Since $PB = PC$, we have $PM = \frac{1}{2}PB = \frac{1}{2}PC = CM$. This implies that triangle $CPM$ is isosceles with $CM = PM = \frac{1}{2}PC$. 

Since $O$ is the circumcenter of $\Gamma$, we have $OM = R$. Since $OM = R$ and $PM = \frac{1}{2}PC$, we have $OP = \frac{1}{2}PC$. Since $PA = PO = R$, we have $AO = R$. 

Now, consider triangle $AOC$. Since $AO = OC = R$ and $\angle AOC = 2\alpha$, we have $\angle OAC = \angle OCA = 90^{\circ} - \alpha$. 

Since $AO = OC = R$ and $\angle OAC = \angle OCA = 90^{\circ} - \alpha$, we have $\angle ACO = \alpha$. 

Now, consider triangle $CPD$. Since $\angle CPD = 2\alpha$ and $\angle CDP = \angle PDC = \alpha$, we have $\angle DCP = 90^{\circ} - \alpha$. 

Since $\angle DCP = 90^{\circ} - \alpha$ and $\angle ACO = 90^{\circ} - \alpha$, we have $\angle DCP = \angle ACO$. 

Since $\angle DCP = \angle ACO$ and $CP = CA$, we have $DP = DA$. 

Similarly, we can show that $EP = EA$. 

Therefore, $P$ is the incenter of triangle $CDE$ since $DP = DA$, $EP = EA$, and $\angle DCP = \angle ACO$.
