Write  

  \(\displaystyle c_n(x)=\#\{\,j\le n\mid a_j=x\}\)       (the number of occurrences of the integer \(x\) among the first \(n\) terms).

For every \(n>N\)

  \(\displaystyle a_n=c_{\,n-1}(a_{n-1}).\tag{1}\)

Step 1.  1 occurs infinitely many times  
———————————————————————————  
Assume that 1 occurs only finitely many times; say no term beyond the index \(M\ge N\) equals 1.  
Because every first–time (new) value is immediately followed by 1 (by (1) the
first occurrence of a value \(v\) is followed by \(a_{n+1}=c_{\,n}(v)=1\)),
no new value can appear after \(M-1\).
Hence from the index \(M\) on only a finite set  

  \(S=\{a_M,a_{M+1},a_{M+2},\dots\}\)

of positive integers is used.  
Choose a value \(s\in S\) that occurs infinitely many times.
Let \(k\) be larger than all elements of \(S\).
Because \(s\) occurs infinitely often, pick an index \(n\ge M\) at which
\(s\) appears for the \(k\)-th time.  
Then \(a_{n}=s\) and, by (1),

  \(a_{n+1}=c_{\,n}(s)=k\;>\;\max S\),

contradicting \(a_{n+1}\in S\).
Thus 1 must occur infinitely often in the full sequence.

Step 2. 1 occurs infinitely often in at least one parity  
—————————————————————————————————  
The infinitely many 1’s occupy either infinitely many odd places or
infinitely many even places (or both).  
Without loss of generality assume

  \(a_{2m}=1\) for infinitely many indices \(m\).

Step 3.  From some point on every even term equals 1  
————————————————————————————————  
Fix the smallest index \(m_0\) with \(a_{2m_0}=1\).
We prove by induction on \(m\ge m_0\) that

  \(a_{2m}=1.\)

Base \(m=m_0\) is true by definition.  
Inductive step:
suppose \(a_{2m}=1\) with \(m\ge m_0\).
Let  

  \(t=a_{2m+1}=c_{\,2m}(1)\)

be the number of 1’s among the first \(2m\) terms (formula (1)).
Because a new 1 just appeared at position \(2m\), this count increased by
one; consequently \(t\) has never appeared before and is a
new value.
Therefore the next term is

  \(a_{2m+2}=c_{\,2m+1}(t)=1.\)

Hence \(a_{2(m+1)}=1\) and the induction is complete.

Thus all even–indexed terms from position \(2m_0\) onward are equal to 1;
the even subsequence  

  \(a_2,\,a_4,\,a_6,\,\dots\)

is ultimately the constant sequence 1, i.e. it is periodic (with period 1).

Step 4.  Conclusion  
—————————  
We have shown that if the even positions contain infinitely many 1’s, the
even subsequence is eventually periodic.  
The same argument, applied verbatim to the odd positions, shows that
if instead the odd positions contain infinitely many 1’s, the odd
subsequence is eventually periodic.

Because 1 appears infinitely many times in the whole sequence,
at least one of the two parity classes contains infinitely many 1’s;
hence at least one of the subsequences  

  \(a_1,\,a_3,\,a_5,\dots\quad\text{or}\quad a_2,\,a_4,\,a_6,\dots\)

is eventually periodic.

This completes the proof.