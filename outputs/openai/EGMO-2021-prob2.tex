Let  

P(x ,y) : f ( x f(x) + y ) = f (y) + x²    (∀ x ,y ∈ ℚ)        (1)

be the assertion of the problem.  
All variables below are rational unless something else is said.

--------------------------------------------------------------------
1.  f(0)=0
--------------------------------------------------------------------
Put x = 0 in (1):

 f(y + f(0)) = f(y)    (∀ y)                                   (2)

Thus f is periodic with period c := f(0).  
Assume c ≠ 0.  
Because of the periodicity, the set { f(t) ; 0 ≤ t < |c| } already contains the whole
range of f, hence f is bounded.

But from (1) with y = 0 we have

 f(x f(x)) = c + x².                             (3)

The right–hand side can be made arbitrarily large just by choosing x with large
absolute value, contradicting the boundedness of f.  
Therefore c = 0.

So

 f(0)=0 and f(y+0)=f(y) gives no other information.             (4)


--------------------------------------------------------------------
2.  A useful translation identity
--------------------------------------------------------------------
Because f(0)=0, (1) with y := 0 reads

 f(x f(x)) = x²                                                  (5)

Put y := −x f(x) in (1):

 f(−x f(x)) = −x².                                              (6)

Subtracting (6) from (5) yields

 f(x f(x)) − f(−x f(x)) = 2x².                                  (7)

Now note that (1) may be rewritten as

 f(y + x f(x)) − f(y) = x²    (∀ x ,y).                      (8)

Hence, for every rational s belonging to the set

 T := { x f(x) ; x ∈ ℚ }                                        (9)

the difference f(y+s) − f(y) is independent of y.  
Define

 h(s) := f(y+s) − f(y)    (any y, since the RHS is y–independent).     (10)

For s = x f(x) we have h(s)=x² by (8).  
Moreover, for any s₁,s₂ ∈ T,

 h(s₁+s₂)=h(s₁)+h(s₂)                                          (11)

because

  f(y+s₁+s₂)−f(y) = [f(y+s₁+s₂)−f(y+s₂)]+[f(y+s₂)−f(y)]
                   = h(s₁)+h(s₂).

So h is additive on the subgroup T of ℚ.


--------------------------------------------------------------------
3.  The set T is all of ℚ
--------------------------------------------------------------------
Take an arbitrary rational q.  
Because of (5) and (6) we have both

  +q = ( √|q| ) f( √|q| ) and −q = ( −√|q| ) f( −√|q| ),

so ±q ∈ T.  
Consequently T = ℚ.

Therefore h is an additive function on the whole field ℚ and still satisfies

 h(x f(x)) = x²  (= h( (x f(x)) ) ).                          (12)

--------------------------------------------------------------------
4.  Determination of h
--------------------------------------------------------------------
Fix any rational t.  
Because T = ℚ, there is an x with t = x f(x).  
Then by (12),

 h(t)=x².

But t = x f(x) implies x = t / f(x).  
Insert this into the last equation:

 h(t) = (t / f(x))².                                            (13)

Here the right–hand side depends on x only through the ratio t / f(x),
hence f(x) must be a constant multiple of x; otherwise for a fixed
t different choices of x (having the same product t) would give different
values for h(t), contradicting (13).

Thus there is a constant ε ∈ ℚ, ε ≠ 0, with

 f(x) = ε x    (∀ x).                                         (14)

--------------------------------------------------------------------
5.  Identification of ε
--------------------------------------------------------------------
Insert (14) into the original equation (1):

   ε( x·εx + y ) = ε y + x²
   ε² x² + ε y   = ε y + x²
             ε² = 1.                                             (15)

Hence ε = 1 or ε = −1.

--------------------------------------------------------------------
6.  Solutions
--------------------------------------------------------------------
Therefore the only functions ℚ→ℚ satisfying (1) are

 f(x) =  x  and f(x) = −x  for all x ∈ ℚ.

Both indeed fulfil the given relation, so they are the complete list of solutions.