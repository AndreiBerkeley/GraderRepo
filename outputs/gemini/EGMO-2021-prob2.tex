Here is a complete and rigorous, step-by-step solution to the problem.

### Problem
Find the value of the sum:
$$S = \sum_{k=1}^{3599} \sin^3\left(\frac{k\pi}{1800}\right)$$

### Method 1: Using Symmetry

This method is the most direct and elegant. It relies on the symmetry properties of the sine function.

**Step 1: Define the general term**
Let the general term of the summation be denoted by $f(k)$:
$$f(k) = \sin^3\left(\frac{k\pi}{1800}\right)$$

**Step 2: Analyze the symmetry of the terms**
We will investigate the relationship between the terms $f(k)$ and $f(3600-k)$. The number $3600$ is chosen because the argument of the sine function, $\frac{k\pi}{1800}$, completes a full cycle of $2\pi$ as $k$ goes from $0$ to $3600$.

Let's evaluate $f(3600-k)$:
$$f(3600-k) = \sin^3\left(\frac{(3600-k)\pi}{1800}\right) = \sin^3\left(\frac{3600\pi}{1800} - \frac{k\pi}{1800}\right) = \sin^3\left(2\pi - \frac{k\pi}{1800}\right)$$
We use the trigonometric identity $\sin(2\pi - x) = -\sin(x)$. Applying this identity, we get:
$$f(3600-k) = \left[\sin\left(2\pi - \frac{k\pi}{1800}\right)\right]^3 = \left[-\sin\left(\frac{k\pi}{1800}\right)\right]^3$$
$$f(3600-k) = -\sin^3\left(\frac{k\pi}{1800}\right) = -f(k)$$
This gives us a crucial symmetry property:
$$f(k) + f(3600-k) = 0$$

**Step 3: Pair the terms in the sum**
We can now pair the terms in the sum $S$ from opposite ends. The sum is from $k=1$ to $k=3599$.
$$S = f(1) + f(2) + \dots + f(1799) + f(1800) + f(1801) + \dots + f(3598) + f(3599)$$
Let's group the terms into pairs $(f(k), f(3600-k))$:
$$S = \big[f(1) + f(3599)\big] + \big[f(2) + f(3598)\big] + \dots + \big[f(1799) + f(1801)\big] + f(1800)$$
Each of these pairs sums to zero due to the symmetry property we found:
$$f(k) + f(3600-k) = 0 \quad \text{for } k=1, 2, \dots, 1799$$
Thus, the sum simplifies to:
$$S = \sum_{k=1}^{1799} \big[f(k) + f(3600-k)\big] + f(1800) = \sum_{k=1}^{1799} 0 + f(1800) = f(1800)$$

**Step 4: Evaluate the middle term**
The only term that was not part of a pair is the middle term, $f(1800)$. Let's evaluate it:
$$f(1800) = \sin^3\left(\frac{1800\pi}{1800}\right) = \sin^3(\pi)$$
Since $\sin(\pi) = 0$, we have:
$$f(1800) = 0^3 = 0$$

**Step 5: Final Conclusion**
Combining our results, we find that the sum is zero.
$$S = 0$$

---

### Method 2: Using Trigonometric Identity (Linearization)

This method involves using a trigonometric identity to reduce the power of the sine function.

**Step 1: Linearize the `sin^3` term**
We use the triple angle identity for sine: $\sin(3x) = 3\sin(x) - 4\sin^3(x)$.
Rearranging this identity gives us a formula for $\sin^3(x)$:
$$\sin^3(x) = \frac{3\sin(x) - \sin(3x)}{4}$$

**Step 2: Substitute the identity into the sum**
Let $x_k = \frac{k\pi}{1800}$. The sum $S$ becomes:
$$S = \sum_{k=1}^{3599} \frac{1}{4}\left[3\sin\left(\frac{k\pi}{1800}\right) - \sin\left(\frac{3k\pi}{1800}\right)\right]$$
We can split this into two separate sums:
$$S = \frac{3}{4}\sum_{k=1}^{3599} \sin\left(\frac{k\pi}{1800}\right) - \frac{1}{4}\sum_{k=1}^{3599} \sin\left(\frac{k\pi}{600}\right)$$
Let's call these sums $S_1$ and $S_2$:
$$S_1 = \sum_{k=1}^{3599} \sin\left(\frac{k\pi}{1800}\right) \quad \text{and} \quad S_2 = \sum_{k=1}^{3599} \sin\left(\frac{k\pi}{600}\right)$$
So, $S = \frac{3}{4}S_1 - \frac{1}{4}S_2$.

**Step 3: Evaluate the sum S1**
We can use the same symmetry argument as in Method 1. Let $g(k) = \sin\left(\frac{k\pi}{1800}\right)$.
Then $g(3600-k) = \sin\left(2\pi - \frac{k\pi}{1800}\right) = -\sin\left(\frac{k\pi}{1800}\right) = -g(k)$.
$$S_1 = \sum_{k=1}^{3599} g(k) = \sum_{k=1}^{1799} [g(k) + g(3600-k)] + g(1800) = \sum_{k=1}^{1799} 0 + \sin(\pi) = 0$$
Thus, $S_1 = 0$.

**Step 4: Evaluate the sum S2**
Let $h(k) = \sin\left(\frac{k\pi}{600}\right)$. The period of $h(k)$ with respect to integer $k$ is found by setting the argument change to $2\pi$: $\frac{N\pi}{600} = 2\pi \implies N=1200$. So, $h(k+1200) = h(k)$.
The sum for $S_2$ runs from $k=1$ to $3599$. This is nearly three full periods of $1200$.
$$S_2 = \sum_{k=1}^{3599} h(k) = \left(\sum_{k=1}^{1200} h(k)\right) + \left(\sum_{k=1201}^{2400} h(k)\right) + \left(\sum_{k=2401}^{3599} h(k)\right)$$
First, let's evaluate the sum over one full period:
$$\sum_{k=1}^{1200} h(k) = \sum_{k=1}^{1199} \sin\left(\frac{k\pi}{600}\right) + \sin\left(\frac{1200\pi}{600}\right)$$
The last term is $\sin(2\pi) = 0$. The sum $\sum_{k=1}^{1199} \sin\left(\frac{k\pi}{600}\right)$ is zero by the same symmetry pairing argument (pairing $k$ with $1200-k$; the middle term is $k=600$, for which $\sin(\pi)=0$). Thus, $\sum_{k=1}^{1200} h(k) = 0$.

The second block is $\sum_{k=1201}^{2400} h(k) = \sum_{j=1}^{1200} h(j+1200) = \sum_{j=1}^{1200} h(j) = 0$.
The third block is $\sum_{k=2401}^{3599} h(k) = \sum_{j=1}^{1199} h(j+2400) = \sum_{j=1}^{1199} h(j) = 0$.

Therefore, $S_2 = 0 + 0 + 0 = 0$.

**Step 5: Final Conclusion**
Substituting the values of $S_1$ and $S_2$ back into the expression for $S$:
$$S = \frac{3}{4}(0) - \frac{1}{4}(0) = 0$$

Both methods rigorously show that the value of the sum is 0.