This is a classic and elegant geometry problem that can be solved using properties of isosceles triangles derived from parallel lines or, more swiftly, via homothety. We will first demonstrate that the point P is, in fact, the same as the point A.

### Step 1: Proving A is the midpoint of MN

We will show that $AM = AN$. Since M, A, and N are collinear (all lie on line $l$), this will prove that A is the midpoint of the segment MN.

**Method 1: Using Angle Properties**

Let the angles of $\triangle ABC$ be $\angle A = \alpha$, $\angle B = \beta$, and $\angle C = \gamma$.
Since I is the incenter, the segments BD and BF are tangents from vertex B to the incircle. Thus, $BD = BF$, which implies that $\triangle BDF$ is an isosceles triangle.
The angle at vertex B is $\angle FBD = \beta$. The base angles of $\triangle BDF$ are $\angle BDF = \angle BFD = \frac{180^\circ - \beta}{2} = 90^\circ - \frac{\beta}{2}$.

The line $l$ passes through A and is parallel to BC. The line DF intersects these two parallel lines, acting as a transversal.
Let's analyze the angles in $\triangle AFN$. The vertices are A, F, N.
1.  **Angle at A ($\angle FAN$):** The line segment AF lies on the line AB. The segment AN lies on the line $l$. The angle $\angle FAN$ is the angle between the line AB and the line $l$. Since $l \parallel BC$, the angle between $l$ and the transversal AB is equal to the alternate interior angle $\angle ABC = \beta$. Thus, $\angle FAN = \beta$.
2.  **Angle at N ($\angle ANF$):** This is the angle between the line $l$ and the line DF. As corresponding angles between parallel lines ($l$ and BC) with transversal DF, this angle must be equal to the angle between BC and DF. The angle between the line BC and the line segment DF is $\angle BDF = 90^\circ - \frac{\beta}{2}$. Thus, $\angle ANF = 90^\circ - \frac{\beta}{2}$.
    *(Note: One must be careful about whether the angle is acute or obtuse. Based on a standard diagram, the angle is acute, matching the corresponding angle directly.)*

Now we find the third angle of $\triangle AFN$:
$\angle AFN = 180^\circ - \angle FAN - \angle ANF = 180^\circ - \beta - (90^\circ - \frac{\beta}{2}) = 90^\circ - \frac{\beta}{2}$.

Since $\angle AFN = \angle ANF = 90^\circ - \frac{\beta}{2}$, the triangle $\triangle AFN$ is isosceles with the sides opposite to equal angles being equal in length.
Therefore, $AN = AF$.

By a symmetrical argument for $\triangle AEM$:
The segments CD and CE are tangents from vertex C to the incircle, so $CD = CE$. $\triangle CDE$ is an isosceles triangle.
$\angle ECD = \gamma$, so the base angles are $\angle CDE = \angle CED = \frac{180^\circ - \gamma}{2} = 90^\circ - \frac{\gamma}{2}$.

Now, consider the angles in $\triangle AEM$:
1.  **Angle at A ($\angle EAM$):** This is the angle between the line AC and the line $l$. Since $l \parallel BC$, $\angle EAM = \angle ACB = \gamma$ (alternate interior angles).
2.  **Angle at M ($\angle AME$):** This is the angle between the line $l$ and the line DE. As corresponding angles with transversal DE, $\angle AME = \angle CDE = 90^\circ - \frac{\gamma}{2}$.

The third angle of $\triangle AEM$ is:
$\angle AEM = 180^\circ - \angle EAM - \angle AME = 180^\circ - \gamma - (90^\circ - \frac{\gamma}{2}) = 90^\circ - \frac{\gamma}{2}$.

Since $\angle AEM = \angle AME = 90^\circ - \frac{\gamma}{2}$, the triangle $\triangle AEM$ is isosceles with $AM = AE$.

Finally, AF and AE are tangents from the same vertex A to the incircle. Thus, $AF = AE$.
Combining our results, we have $AN = AF = AE = AM$.
So, $AN = AM$. Since M and N both lie on the line $l$ and A is also on $l$, A must be the midpoint of the segment MN.

**Method 2: Using Homothety (a more direct approach)**

Let's consider a homothety (dilation) $H_F$ centered at point F. Since F is the point of tangency of the incircle with the side AB, F lies between A and B. We choose the homothety that maps point B to point A. The ratio of this homothety is $k = -AF/BF$. The negative sign indicates that the center F is between the point B and its image A.

-   The image of point B under $H_F$ is A.
-   The image of the line BC is a line parallel to BC passing through the image of B, which is A. This is precisely the line $l$.
-   The image of the line DF is the line DF itself, as it passes through the center of homothety F.
-   The point D is the intersection of the line BC and the line DF. Its image, $H_F(D)$, must be the intersection of the images of these lines. The image of BC is $l$ and the image of DF is DF. Their intersection is the point N. Therefore, $H_F(D) = N$.

Now, because homothety maps triangles to similar triangles, we have that $\triangle FBD$ is mapped to $\triangle FAN$. Thus, $\triangle FAN \sim \triangle FBD$.

In $\triangle FBD$, the sides BF and BD are tangents from B to the incircle, so $BF = BD$. This means $\triangle FBD$ is an isosceles triangle.
Since $\triangle FAN$ is similar to $\triangle FBD$, it must also be an isosceles triangle. The sides corresponding to BF and BD are AF and AN, respectively. Therefore, we must have $AF = AN$.

Similarly, consider a homothety $H_E$ centered at E with ratio $k' = -AE/CE$ that maps C to A.
-   $H_E(C) = A$.
-   The image of line BC is line $l$.
-   The image of line DE is line DE itself.
-   The image of D = BC $\cap$ DE is M = $l$ $\cap$ DE.
-   Thus, $\triangle EAM \sim \triangle ECD$.

In $\triangle ECD$, $CE = CD$ (tangents from C), so it is an isosceles triangle.
Since $\triangle EAM$ is similar to $\triangle ECD$, it must also be isosceles with $AE=AM$.

As before, we know $AF=AE$. So we have $AN = AF = AE = AM$, which means $AN = AM$.
Therefore, A is the midpoint of MN.

### Step 2: Conclusion

The problem states that P is the midpoint of the segment MN. From our derivation in Step 1, we have proven that A is the midpoint of MN. Therefore, the point P must be the same as the point A, i.e., $P=A$.

Now we are asked to prove that A is the incenter of $\triangle BPC$.
Substituting $P=A$, the problem becomes: **"Prove that A is the incenter of $\triangle BAC$."**

A vertex of a non-degenerate triangle cannot be its incenter. The incenter is a point in the interior of the triangle, equidistant from the three sides. A vertex, say A, is on two of the sides (AB and AC), so its distance to these sides is 0. Its distance to the third side (BC) is the altitude from A, $h_a$, which is non-zero for a non-degenerate triangle. Thus, A cannot be the incenter of $\triangle BAC$.

This leads to a paradox. The geometric construction and premises of the problem logically imply that $P=A$. However, the conclusion we are asked to prove is false if we substitute $P=A$. This suggests that there might be an error in the problem statement as it is given. The substantive part of the problem, and likely the intended challenge, is to prove that A is the midpoint of MN.

Assuming the problem intended to ask for a proof of a derivable fact, a plausible intended question would have been simply **"Prove that A is the midpoint of MN"**.