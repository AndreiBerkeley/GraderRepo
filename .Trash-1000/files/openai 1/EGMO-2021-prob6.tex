Let's denote the given equation as
\[
\sum_{i=1}^{m} \left\lfloor\frac{m}{i}\right\rfloor = n^2 + a.
\]
We know that for any positive integer \( m \), the sum
\[
\sum_{i=1}^{m} \left\lfloor\frac{m}{i}\right\rfloor
\]
is the number of lattice points in the first quadrant of the graph of the function \( y = \frac{m}{x} \) for \( 1 \leq x \leq m \). This is because \( \left\lfloor\frac{m}{i}\right\rfloor \) is the number of lattice points on the line \( x = i \) below the graph of \( y = \frac{m}{x} \).

Now, let's consider the function \( f(x) = \frac{m}{x} \) for \( x \geq 1 \). The number of lattice points in the first quadrant of the graph of this function is equal to the sum
\[
\sum_{i=1}^{m} \left\lfloor\frac{m}{i}\right\rfloor.
\]
We know that the number of lattice points in the first quadrant of the graph of this function is at most \( m^2 \) because the graph is contained in the rectangle with vertices at \( (1, m) \), \( (m, 1) \), \( (m, 0) \), and \( (0, m) \).

Therefore, we have
\[
\sum_{i=1}^{m} \left\lfloor\frac{m}{i}\right\rfloor \leq m^2.
\]
This implies that
\[
n^2 + a \leq m^2.
\]
Since we want to find a nonnegative integer \( a \) for which the equation has more than one million different solutions, we need to find the maximum number of solutions for a given \( a \).

Let's consider the case when \( a = 0 \). In this case, we have
\[
n^2 \leq m^2.
\]
The number of solutions to this inequality is given by the number of lattice points in the first quadrant of the graph of \( y = x \) for \( 1 \leq x \leq m \), which is \( m \). Therefore, for \( a = 0 \), the equation has at most \( m \) solutions.

To find the maximum number of solutions, we need to find the largest possible value of \( a \) such that the equation has more than one million different solutions. Let's denote this value as \( a_{\text{max}} \).

Since we want to maximize the number of solutions, we need to minimize the value of \( m \). The smallest possible value of \( m \) is 1, which gives us \( n^2 \leq 1 \). This has only one solution, \( n = 1 \), and \( m = 1 \).

Therefore, for \( a = 0 \), we have one solution. To find the maximum value of \( a \) for which the equation has more than one million different solutions, we need to find the largest possible value of \( a \) such that the equation has more than one million solutions when \( m = 1 \).

For \( m = 1 \), the equation becomes
\[
\left\lfloor\frac{1}{1}\right\rfloor = n^2 + a,
\]
which simplifies to
\[
1 = n^2 + a.
\]
Since \( n \) is a positive integer, the smallest possible value of \( n \) is 1, which gives us \( a = 0 \). This gives us one solution.

Therefore, the equation has at most one solution for any nonnegative integer \( a \). Since we want to find a nonnegative integer \( a \) for which the equation has more than one million different solutions, such an \( a \) does not exist.
