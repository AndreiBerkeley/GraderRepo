Let's consider the set $D$ after the first recoloring. Let $d_0$ be the smallest positive real number in $D$. Since $d_0$ is in $D$, there exist two points of the same color at distance $d_0$ apart. After the first recoloring, these two points will be colored red. Now, all the distances less than $d_0$ are not in $D$ and will be colored blue.

Consider any distance $d > d_0$. If there are two points of the same color at distance $d$ apart, then $d$ will be in $D$ after the first recoloring. Otherwise, $d$ will not be in $D$ after the first recoloring. This is because if there are two points at distance $d$ apart with different colors, then one of them must be at a distance less than $d_0$ from one of the points at distance $d_0$ apart, which is not possible as all distances less than $d_0$ are not in $D$ after the first recoloring.

Therefore, after the first recoloring, all distances less than $d_0$ are not in $D$ and all distances greater than or equal to $d_0$ are in $D$. This means that $d_0$ is the smallest positive real number in $D$ after the first recoloring.

Now, if we iterate this process, we will keep identifying the smallest positive real number in $D$ and recoloring the points accordingly. At each step, the smallest positive real number in $D$ will be colored red, and all other distances will be colored blue. Since the set of positive real numbers is well-ordered, this process will eventually color all the positive real numbers red after a finite number of steps.

Therefore, after a finite number of steps, we will end up with all the numbers red.
