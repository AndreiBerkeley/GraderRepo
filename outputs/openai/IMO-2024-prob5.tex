Answer:  2023  

Explanation  

1.  A strategy that always works in at most 2023 attempts  

Label the columns \(1,2,\dots ,2023\).  
On his \(j\)-th attempt (\(j=1,2,\dots ,2023\))

 • Turbo starts in the lowest–numbered column that he has never tried before (that
  is, in column \(j\)).  
 • He walks straight down that column until either  
  – he meets a monster (the attempt ends in defeat and he has learned that this
     column contains a monster), or  
  – he reaches the last row (the attempt ends in victory).

Because there are exactly \(2022\) monsters and they are all in different
columns, at most \(2022\) of the \(2023\) columns can harbour a monster.
Consequently the column tried on or before the \(2023^{\text{rd}}\) attempt is
certain to be the unique completely monster–free column, so Turbo has reached
the bottom row by that time.  Hence

  Turbo can always succeed in at most \(2023\) attempts.       \((\ast )\)



2.  Why no strategy can guarantee success sooner  

Assume, to obtain a contradiction, that some strategy would always reach the
bottom row in at most \(2022\) attempts.  
Fix such a strategy and look at the information Turbo possesses after he has
failed \(k\,(<2022)\) times:

 • He has discovered the positions of exactly \(k\) monsters, one per failed
  attempt.  
 • Therefore he knows precisely \(k\) columns that contain monsters.  
 • Let
  \[
     S_k:=\{1,2,\dots ,2023\}\setminus\{\text{those }k\text{ columns}\}.
  \]
  These \(|S_k|=2023-k\ge 2\) columns are still “possible” locations for the
  unique monster-free column.

Look now at how the given strategy is going to conduct the next (namely, the
\((k+1)\)-st) attempt.  Long before the monsters are really placed we can
simulate that attempt and obtain the finite sequence of cells

\[
     P=(p_1,p_2,\dots ,p_m),\qquad p_t=(r_t,c_t),
\]

that Turbo will step on **provided he never meets an undiscovered
monster**.  (The sequence is well defined because the strategy is a fixed set
of rules depending only on previous experiences.)

Because \(|S_k|\ge 2\), pick two different columns  
\(a,b\in S_k\).

Case 1.  The path \(P\) visits column \(a\) in some as-yet-unrevealed row
\(r\).  
Construct a monster placement still consistent with all earlier observations as
follows:

 • Put the (so far unknown) monster of row \(r\) at the very cell \((r,a)=p_t\)
  that Turbo is going to step on.  
 • Put each of the remaining unknown monsters in a **distinct** column chosen
  among the other columns of \(S_k\setminus\{a\}\); this is possible because
  there are at least as many still-free columns as still-empty rows.  

All conditions of the original problem are fulfilled and Turbo
inevitably hits the monster in \((r,a)\) on his \((k+1)\)-st attempt.  
Thus he certainly cannot reach the bottom row in \(k+1\) attempts.

Case 2.  The path \(P\) avoids column \(a\) completely.
Do the same construction with column \(b\) instead of \(a\).
Again a legal arrangement of monsters is produced that forces Turbo to be
eaten on the coming attempt.

Hence, **whatever the strategy is**, as long as fewer than \(2022\) monsters
have been located there is a legal way to place the remaining monsters that
defeats the very next attempt.  Therefore Turbo must first
discover all \(2022\) monster columns,
and can succeed only on the subsequent try.

Consequently

  No strategy can guarantee success in fewer than \(2023\) attempts.       \((\dagger )\)



3.  Conclusion  

Combining \((\ast )\) and \((\dagger )\) we obtain that the smallest integer
\(n\) with the desired property is

\[
\boxed{\,n=2023\,}.
\]