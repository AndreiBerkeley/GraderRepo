Let $X$ be the intersection of $BC$ and $AI$. Since $I$ is the incenter of $\triangle ABC$, we have $IX \perp BC$. Let $Y$ be the intersection of $AI$ and the circumcircle of $\triangle ABC$. Since $D$ is on the circumcircle of $\triangle ABC$, we have $\angle AYD = \angle AXD = 90^\circ$. Therefore, $AXDY$ is cyclic.

Since $I$ is the incenter of $\triangle ABC$, we have $AI$ bisects $\angle BAC$. Therefore, $\angle BAI = \angle IAC$. Since $AXDY$ is cyclic, we have $\angle AYX = \angle ADX$. Since $I$ is the incenter of $\triangle ABC$, we have $IX \perp BC$, so $\angle ADX = \angle AYX = 90^\circ - \angle AYX = 90^\circ - \angle AYD = \angle ADY$. Therefore, $\angle BAI = \angle IAC = \angle ADY$, which implies that $AD$ is parallel to $BC$.

Now, consider the circumcircle of $\triangle DII_a$. Since $AD$ is parallel to $BC$, we have $\angle DII_a = \angle DAI = \angle BAD$. Similarly, considering the circumcircle of $\triangle DI_bI_c$, we have $\angle DI_bI_c = \angle DAI = \angle CAD$.

Now, let $Z$ be the intersection of $AD$ and $BC$. Since $AD$ is parallel to $BC$, we have $\angle DZI = \angle DAI = \angle CAD$. Since $I$ is the incenter of $\triangle ABC$, we have $\angle IZD = \angle IAD = \angle CAD$. Therefore, $\triangle IZD$ is isosceles, which implies that $IZ = ID$.

Now, consider the circumcircle of $\triangle DII_a$. Since $IZ = ID$, we have $\triangle IZD$ is isosceles, which implies that $\angle IZD = \angle IDZ = \angle DII_a$. Therefore, $\angle DII_a = \angle DZI = \angle CAD$. Similarly, considering the circumcircle of $\triangle DI_bI_c$, we have $\angle DI_bI_c = \angle DAI = \angle BAD$.

Since $\angle DII_a = \angle CAD$ and $\angle DI_bI_c = \angle BAD$, we have $\angle CAD = \angle BAD$. Therefore, $\angle BAD = \angle CAD$, which implies that $\angle BAD = \angle EAC$.
