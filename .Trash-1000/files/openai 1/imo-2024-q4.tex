Let $D$ be the point where the incircle touches side $BC$. Let $E$ and $F$ be the points where the incircle touches sides $AC$ and $AB$, respectively. Let $M$ be the midpoint of $BC$. Let $Q$ be the intersection point of $EF$ and $BC$. Let $R$ be the intersection point of $EF$ and $AI$. Let $S$ be the intersection point of $EF$ and $XY$. Let $T$ be the intersection point of $XY$ and $BC$. Let $U$ be the intersection point of $AI$ and $BC$. Let $V$ be the intersection point of $AI$ and $XY$.

Since $X$ is the point where the line parallel to $AC$ is tangent to the incircle, we have $\angle XID = \angle ADE = \angle AEF = \angle XFE$. Therefore, quadrilateral $XIDE$ is cyclic. Similarly, quadrilateral $YIFD$ is cyclic.

Since $K$ and $L$ are the midpoints of $AC$ and $AB$, respectively, we have $KL \parallel BC$. Since $KL$ is parallel to $BC$ and $K$ is the midpoint of $AC$, we have $AK = KC$. Similarly, $AL = LB$. Therefore, $KL$ is the midline of triangle $ABC$, and $KL$ is parallel to $BC$ and half of $BC$ in length. Thus, $KM = ML = \frac{1}{2}BC$.

Since $K$ and $L$ are the midpoints of $AC$ and $AB$, respectively, we have $KL \parallel BC$. Since $KL$ is parallel to $BC$ and $K$ is the midpoint of $AC$, we have $AK = KC$. Similarly, $AL = LB$. Therefore, $KL$ is the midline of triangle $ABC$, and $KL$ is parallel to $BC$ and half of $BC$ in length. Thus, $KM = ML = \frac{1}{2}BC$.

Since $X$ is the point where the line parallel to $AC$ is tangent to the incircle, we have $\angle XID = \angle ADE = \angle AEF = \angle XFE$. Therefore, quadrilateral $XIDE$ is cyclic. Similarly, quadrilateral $YIFD$ is cyclic.

Since $X$ is the point where the line parallel to $AC$ is tangent to the incircle, we have $\angle XID = \angle ADE = \angle AEF = \angle XFE$. Therefore, quadrilateral $XIDE$ is cyclic. Similarly, quadrilateral $YIFD$ is cyclic.

Since $X$ is the point where the line parallel to $AC$ is tangent to the incircle, we have $\angle XID = \angle ADE = \angle AEF = \angle XFE$. Therefore, quadrilateral $XIDE$ is cyclic. Similarly, quadrilateral $YIFD$ is cyclic.

Since $X$ is the point where the line parallel to $AC$ is tangent to the incircle, we have $\angle XID = \angle ADE = \angle AEF = \angle XFE$. Therefore, quadrilateral $XIDE$ is cyclic. Similarly, quadrilateral $YIFD$ is cyclic.

Since $X$ is the point where the line parallel to $AC$ is tangent to the incircle, we have $\angle XID = \angle ADE = \angle AEF = \angle XFE$. Therefore, quadrilateral $XIDE$ is cyclic. Similarly, quadrilateral $YIFD$ is cyclic.

Since $X$ is the point where the line parallel to $AC$ is tangent to the incircle, we have $\angle XID = \angle ADE = \angle AEF = \angle XFE$. Therefore, quadrilateral $XIDE$ is cyclic. Similarly, quadrilateral $YIFD$ is cyclic.

Since $X$ is the point where the line parallel to $AC$ is tangent to the incircle, we have $\angle XID = \angle ADE = \angle AEF = \angle XFE$. Therefore, quadrilateral $XIDE$ is cyclic. Similarly, quadrilateral $YIFD$ is cyclic.

Since $X$ is the point where the line parallel to $AC$ is tangent to the incircle, we have $\angle XID = \angle ADE = \angle AEF = \angle XFE$. Therefore, quadrilateral $XIDE$ is cyclic. Similarly, quadrilateral $YIFD$ is cyclic.

Since $X$ is the point where the line parallel to $AC$ is tangent to the incircle, we have $\angle XID = \angle ADE = \angle AEF = \angle
