Let $O$ be the circumcenter of triangle $ABP$. Since $O$ lies on the perpendicular bisector of $AB$, we have $OA = OB$. 

Since $O$ is the circumcenter of triangle $ABP$, we have $OA = OP$. 

Since $M$ is the midpoint of $BC$, we have $AM = \frac{1}{2}AC$. 

Since $P$ is the foot of the perpendicular from $C$ to $AM$, we have $CP = \frac{1}{2}AC$. 

Therefore, $OP = CP$. 

Since $OA = OP$ and $CP = \frac{1}{2}AC$, we have $OA = \frac{1}{2}AC$. 

Since $N$ is the midpoint of $AQ$, we have $AN = \frac{1}{2}AC$. 

Therefore, $OA = AN$. 

Since $OA = AN$ and $OB = BM$, we have $\triangle OAN \cong \triangle OBM$ by SAS congruence. 

Therefore, $\angle OAN = \angle OBM$ and $\angle OAN = \angle OBM$. 

Since $O$ is the circumcenter of triangle $ABP$, we have $\angle OBP = 2\angle OAP$. 

Since $\angle OAN = \angle OBM$ and $\angle OAN = \angle OBM$, we have $\angle OBP = 2\angle OAN$. 

Therefore, $\angle OBP = 2\angle OAN = 2\angle OBM$. 

Since $\angle OBP = 2\angle OBM$, we have $\angle OBM = \frac{1}{2}\angle OBP$. 

Since $OB = BM$, we have $\triangle OBM$ is isosceles. 

Therefore, $\angle OMB = \angle OBM = \frac{1}{2}\angle OBP$. 

Since $\angle OBP = 2\angle OAN$, we have $\angle OMB = \angle OAN$. 

Therefore, $\angle OMB = \angle OAN$. 

Since $\angle OBM = \angle OAN$ and $OB = BM$, we have $\triangle OBM \cong \triangle OAN$ by SAS congruence. 

Therefore, $ON = OM$ and $NB = NA$. 

Since $ON = OM$ and $NB = NA$, we have $NB = NC$. 

Thus, we have shown that $NB = NC$.
