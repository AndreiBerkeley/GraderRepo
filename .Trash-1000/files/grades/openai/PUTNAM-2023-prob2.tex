% Auto-graded by OpenAI
The candidate solution is incorrect.

The first mistake is in the statement "Since \( q(x) \) has degree \( 2n+1 \), it has at most \( 2n+1 \) roots." This is not true in general. The Fundamental Theorem of Algebra states that a polynomial of degree \( n \) has exactly \( n \) roots (counting multiplicity). Therefore, the fact that \( q(x) \) has \( 2n \) distinct roots does not imply that \( q(x) = 0 \) for all \( x \) such that \( p(x) = x^{2n+2} \).

Overall, the candidate solution is not rigorous and does not provide a correct argument for determining the other real numbers \( x \) for which \( p(1/x) = x^2 \). I would rate this solution a 1 out of 5.
