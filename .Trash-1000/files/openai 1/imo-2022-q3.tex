Let $n$ be the number of elements in the set $S$. We will prove that there is at most one way to place the elements of $S$ around the circle such that the given condition is satisfied.

Let $p_1, p_2, \ldots, p_n$ be the odd prime numbers in $S$. We will show that the only possible way to place these primes around the circle is such that $p_i$ is adjacent to $p_{i+1}$ for $1 \leq i \leq n$.

Suppose there exists a valid arrangement of the primes around the circle such that $p_i$ is not adjacent to $p_{i+1}$ for some $i$. Let $p_i$ and $p_{i+1}$ be separated by $m$ other primes. Then, the product of $p_i$ and $p_{i+1}$ must be of the form $x^2 + x + k$ for some positive integer $x$. Since $p_i$ and $p_{i+1}$ are primes, their product is $p_ip_{i+1}$. Thus, we have $p_ip_{i+1} = x^2 + x + k$.

Now, consider the product of $p_i$ with the prime immediately following $p_{i+1}$ in the circle. This product must also be of the form $x^2 + x + k$. However, the prime immediately following $p_{i+1}$ is $p_{i+2}$, and $p_ip_{i+2}$ is not equal to $x^2 + x + k$ since $p_i$ and $p_{i+2}$ are not adjacent in the circle. This leads to a contradiction, and hence, the assumption that $p_i$ is not adjacent to $p_{i+1}$ must be false.

Therefore, the only valid arrangement of the primes around the circle is such that $p_i$ is adjacent to $p_{i+1}$ for all $1 \leq i \leq n$. This unique arrangement is obtained by placing the primes in a cyclic order around the circle.

Hence, there is at most one way (up to rotation and reflection) to place the elements of $S$ around the circle such that the product of any two neighbors is of the form $x^2 + x + k$ for some positive integer $x$.
