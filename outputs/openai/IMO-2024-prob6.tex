Let  

\[
P(x,y):\qquad  
(\text A)\;f(x+f(y))=f(x)+y\quad\text{or}\quad  
(\text B)\;f\bigl(f(x)+y\bigr)=x+f(y)
\]

be the assertion of the statement.  
Throughout the solution we work only with rational numbers; no other
domain or codomain occurs.

--------------------------------------------------------------------
1.  The value of $f$ at $0$

Put $x=y=0$ in $P$ :

\[
f\bigl(f(0)\bigr)=f(0).\tag{1}
\]

Now put $x=0,\;y=-f(0)$ :

– If $(\text A)$ is true,  
 $f\bigl(f(-f(0))\bigr)=f(0)-f(0)=0$.  
 Because of (1) this gives $f(0)=0$.

– If $(\text B)$ is true,  
 $f\bigl(f(0)-f(0)\bigr)=f(-f(0))$, i.e. $f(0)=f(-f(0))$.  
 Together with (1) it again yields $f(0)=0$.

Hence  

\[
\boxed{\,f(0)=0\,}. \tag{2}
\]

--------------------------------------------------------------------
2.  Surjectivity of $f$

With (2) the alternative $P\bigl(f(0),y\bigr)$ becomes

\[
f\bigl(f(y)\bigr)=y\quad\text{or}\quad f\bigl(f(0)+y\bigr)=f(y).
\]

The second possibility reads $f(y)=f(y)$ and is therefore always true,
while the first one gives $f\bigl(f(y)\bigr)=y$.  
Consequently, for every $y$ there is some argument, namely $f(y)$,
whose image is $y$; hence $f$ is surjective.

--------------------------------------------------------------------
3.  A crucial identity:  $f(-f(x))=-x$

Because of surjectivity we can pick, for any fixed $x$, a rational
$y$ with  

\[
f(y)=-x.
\]

Apply $P(x,y)$.

If $(\text A)$ is true, then

\[
f\bigl(x+f(y)\bigr)=f(x)+y\;\Longrightarrow\;
f(0)=f(x)+y\;\Longrightarrow\;y=-f(x).
\]

If $(\text B)$ is true, then

\[
f\bigl(f(x)+y\bigr)=x+f(y)\;\Longrightarrow\;
f(0)=x-f(x)\;\Longrightarrow\;x=f(x),
\]

and in this case again $y=-f(x)$.  
Hence in either case $y=-f(x)$, and recalling $f(y)=-x$ we obtain  

\[
\boxed{\,f\bigl(-f(x)\bigr)=-x\quad\forall x\in\Bbb Q.\,}\tag{3}
\]

--------------------------------------------------------------------
4.  Oddness and involution

Apply $f$ to both sides of (3) and use (3) once more:

\[
f\bigl(f(-f(x))\bigr)=f(-x)\quad\Longrightarrow\quad
-f(x)=f(-x).
\]

Thus  

\[
\boxed{\,f(-x)=-f(x)\quad\forall x.\,}\tag{4}
\]

Now put $x\mapsto f(x)$ in (4):

\[
f\bigl(-f(f(x))\bigr)=-f(f(x)).
\]

But with (3) the left–hand side equals $-\,f(x)$, so we get  

\[
-\,f(x)=-f(f(x))\;\Longrightarrow\;
\boxed{\,f(f(x))=x\quad\forall x.\,}\tag{5}
\]

Therefore $f$ is an involution and bijective.

--------------------------------------------------------------------
5.  Additivity of $f$

Let $x,y$ be arbitrary.  
Because $f$ is onto, pick $z$ with $f(z)=y$ and apply $P(x,z)$.

• If $(\text A)$ holds,  

\[
f(x+y)=f(x)+z. \tag{6}
\]

• If $(\text B)$ holds,  

\[
f\bigl(f(x)+z\bigr)=x+y.
\]
Using (5) and $z=f(y)$ this becomes  

\[
f(x+y)=f(x)+f(y). \tag{7}
\]

In case (6) we again use $z=f(y)$ (because $f$ is involutive) and get
the same relation (7).
Hence  

\[
\boxed{\,f(x+y)=f(x)+f(y)\quad\forall x,y.\,}\tag{8}
\]

So $f$ is additive on $\Bbb Q$; therefore there exists a constant
$k\in\Bbb Q$ with  

\[
f(x)=k\,x\quad\forall x. \tag{9}
\]

--------------------------------------------------------------------
6.  Determination of the constant $k$

Putting (9) in the involution property (5) gives $k^{2}=1$, hence
$k=\pm1$.  
Thus the only aquaesulian functions are

\[
f(x)=x\qquad\text{and}\qquad f(x)=-x\qquad(x\in\Bbb Q). \tag{10}
\]

--------------------------------------------------------------------
7.  The values $f(r)+f(-r)$

For both functions in (10)

\[
f(r)+f(-r)=\pm r\mp r=0\qquad\text{for every }r\in\Bbb Q.
\]

So every aquaesulian function yields exactly one (and the same)
rational number, namely $0$, of the form $f(r)+f(-r)$.

Consequently there is a universal bound $c=1$, and no smaller integer
is possible.  

--------------------------------------------------------------------
Answer.  
The smallest integer with the required property is  

\[
\boxed{c=1}.
\]