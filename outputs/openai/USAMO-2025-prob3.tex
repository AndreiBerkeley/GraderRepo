Bob (the builder) always has a winning strategy.

Notation.  “Directly similar’’ means similarity obtained by a rotation, a homothety and a translation, but never a reflection.  
For two already fixed, distinct points \(P,Q\) denote by  

\[
\mathcal T_{A,B}\; :\; \Bbb R^{2}\longrightarrow\Bbb R^{2}
\]

the unique direct similarity that sends \(P\mapsto A\) and \(Q\mapsto B\)
(the scale factor and the rotation are determined by the vector \(PQ\mapsto AB\)).
For every point \(X\) we write  

\[
X^{A,B}:=\mathcal T_{A,B}^{-1}(X).
\]

Thus \( \triangle PQR\sim\triangle ABC\)  iff \(C=R^{A,B}\), and   
\( \triangle PQR\sim\triangle BAC\)  iff \(C=(R^{A,B})^{\!*}:=
\mathcal T_{B,A}^{-1}(R)\).

--------------------------------------------------------------------
1.  If  \(S=\Bbb R^{2}\).

Then for every pair of cities \(A,B\) and every third city \(C\) we have
\(\mathcal T_{A,B}(C)\in S\); hence a road is built between every
pair of cities, i.e. the graph of roads is the straight-line drawing of the
complete graph on Bob’s set of cities.

Bob now places four cities forming a convex
quadrilateral whose sides are all longer than \(1\).
No three of them are collinear, so the placement is legal.
Because the drawing is the complete graph,
the two diagonals of that quadrilateral are roads and they intersect.
Therefore condition (ii) (“no two roads cross’’) fails and Bob wins.

--------------------------------------------------------------------
2.  If  \(S\neq\Bbb R^{2}\).

Fix once and for all a point \(R_{0}\notin S\).
Bob will construct his cities inductively so that

(♠)  For every two distinct cities \(A,B\) already placed there is  
a third city \(C\) already placed with
\(\triangle PQR_{0}\) directly similar to \(\triangle ABC\) or \(\triangle BAC\).

If (♠) holds, no pair \((A,B)\) satisfies the rule stated in the
statement (because that rule requires the existence of an \(R\in S\) for
every third city): hence no road will ever be built and the graph of roads
is totally disconnected, making Bob the winner by condition (i).

Inductive construction.  
Suppose after \(n\,(n\ge 0)\) steps Bob has already placed
cities \(C_{1},\dots ,C_{n}\) satisfying

• mutual distances  \(>1\);  

• no three collinear;  

• property (♠) for every pair among them.

List all the finitely many pairs \(\{A,B\}\subset \{C_{1},\dots ,C_{n}\}\)
for which (♠) is still missing, and pick one such pair.
Because \(R_{0}\notin S\), the set
\(\mathcal T_{A,B}^{-1}(R_{0})\)\;(and also
\(\mathcal T_{B,A}^{-1}(R_{0})\)) is a single point, but we are free to
replace \(R_{0}\) by any point of the open set \(\Bbb R^{2}\setminus S\).
That open set is infinite, hence so is its
pre-image under the similarity \(\mathcal T_{A,B}^{-1}\).
Consequently we can choose a point \(C\) that

– lies in \(\mathcal T_{A,B}^{-1}(\Bbb R^{2}\setminus S)\),  
so that \(\triangle PQR_{0}\) is directly similar
to \(\triangle ABC\) or \(\triangle BAC\);  

– is at distance \(>1\) from all \(C_{1},\dots ,C_{n}\);  

– does not belong to any of the finitely many lines determined
by two previously placed cities.

Add that point as \(C_{n+1}\).
All inductive requirements stay satisfied.
Continuing recursively we obtain an infinite, legal set of cities that fulfils (♠).

--------------------------------------------------------------------
3.  Conclusion.

Either Alice chooses \(S=\Bbb R^{2}\), in which case Bob makes four
convex–quadrilateral cities and forces a crossing road, or
\(S\neq\Bbb R^{2}\), in which case Bob follows the inductive scheme above
and forces the road graph to be totally disconnected.
In both cases at least one of Alice’s winning conditions fails,
so Bob has a winning strategy.

Therefore

Bob is the player who always has a winning strategy.