Let's denote the sequences $a_1, a_3, a_5, \ldots$ and $a_2, a_4, a_6, \ldots$ as $b_i$ and $c_i$ respectively.

Since $a_n$ is equal to the number of times $a_{n-1}$ appears in the list $(a_1, a_2, \ldots, a_{n-1})$ for each $n > N$, we have $a_n = \text{count}(a_{n-1})$ for all $n > N$.

Let's consider the sequence $b_i$. Since $b_{2i+1} = a_{2i+1} = \text{count}(a_{2i})$ for all $i > N$, we have $b_{2i+1} = \text{count}(b_{2i})$ for all $i > N$. Similarly, for the sequence $c_i$, we have $c_{2i} = a_{2i} = \text{count}(a_{2i-1})$ for all $i > N$, which implies $c_{2i} = \text{count}(c_{2i-1})$ for all $i > N$.

Now, we will show that one of the sequences $b_i$ or $c_i$ is eventually periodic. Suppose for the sake of contradiction that both sequences are not eventually periodic. This means that for any positive integer $M$, there exists an index $k > M$ such that $b_k \neq b_{k+m}$ for all positive integers $m$ and $c_k \neq c_{k+m}$ for all positive integers $m$.

Consider the set $S = \{b_k, b_{k+1}, b_{k+2}, \ldots\}$. Since $b_i$ is not eventually periodic, $S$ is an infinite set. Similarly, consider the set $T = \{c_k, c_{k+1}, c_{k+2}, \ldots\}$, which is also an infinite set.

Now, since $b_{2i+1} = \text{count}(b_{2i})$ for all $i > N$, the elements of $S$ are non-decreasing. Similarly, since $c_{2i} = \text{count}(c_{2i-1})$ for all $i > N$, the elements of $T$ are non-decreasing.

Since $S$ and $T$ are infinite sets of non-decreasing positive integers, there exist infinitely many distinct elements in both sets. Let $m$ be the smallest positive integer such that $b_{k+m} \notin S$ and $c_{k+m} \notin T$. Then, $b_{k+m} \neq b_{k+n}$ for all positive integers $n < m$ and $c_{k+m} \neq c_{k+n}$ for all positive integers $n < m$.

However, this contradicts the fact that $b_{k+m} = \text{count}(b_{k+m-1})$ and $c_{k+m} = \text{count}(c_{k+m-1})$. Therefore, at least one of the sequences $b_i$ or $c_i$ must be eventually periodic.
