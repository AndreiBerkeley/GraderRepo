Solution outline  
——————————

1.  Think of the circle as the sequence  
   $C_{1},C_{2},\dots ,C_{m}\;(C_{m+1}=C_{1})$ written clockwise.

2.  Fix an order of the people  
   $P_{1},P_{2},\dots ,P_{n}$ (any order will do).

3.  For a starting place $s\in\{1,2,\dots ,m\}$ run the
   following “walk–around–the–circle’’ procedure.

  Step 0 Put the pointer at $C_{s}$.

  Step 1 Give consecutive cupcakes to $P_{1}$, beginning
     with $C_{s}$, until the total score of the cupcakes that
     $P_{1}$ has obtained is at least 1.  
     Let the first cupcake *not* given to $P_{1}$
     be $C_{t_{1}}$; place the pointer there.

  Step 2 Give consecutive cupcakes to $P_{2}$, beginning with
     $C_{t_{1}}$, until $P_{2}$ has accumulated
     score $\ge 1$; call the next cupcake $C_{t_{2}}$.

  \qquad  ⋯  

  Step $n$  Give consecutive cupcakes to $P_{n}$, starting with
     $C_{t_{n-1}}$, until $P_{n}$ has score $\ge1$.
     Stop at the first cupcake that is *not* given to $P_{n}$.
     Denote that stopping place by $C_{\varphi(s)}$.
     Let $L(s)=\varphi(s)-s$ (taken in the integer interval
     $\{1,\dots ,m\}$) be the number of cupcakes that were
     distributed by the procedure that started at $s$.

   The outcome of the procedure that starts at $s$
   is a partition of the *first* $L(s)$ cupcakes met
   after $C_{s}$ into $n$ consecutive
   blocks, the $k$-th block belonging to $P_{k}$ and giving that
   person total score at least 1.

4.  Two key facts about the map $s\mapsto\varphi(s)$.

   A. (Strict monotonicity)  
      If $1\le s\le m$ then, read on the
      infinite line $\dots ,C_{s-1},C_{s},C_{s+1},\dots$,
      the index $\varphi(s+1)$ is *strictly larger* than
      $\varphi(s)$.  
      Indeed, when the procedure begins one cupcake later, the
      first person can never finish sooner; hence every person
      finishes no earlier, so the global stopping place moves
      forward at least one position.

   B. (Bounded motion)  
      Passing the starting point by one cupcake advances
      the stopping point by *at most* one cupcake; consequently
      $\varphi(s+1)=\varphi(s)+1$ for every $s$.

   Together these two statements say that the mapping
      s ↦ (ϕ(s) − s)   (mod m)
   is a permutation of the residue classes
   $0,1,\dots ,m-1$.

5.  As $s$ runs through $1,2,\dots ,m$, one of the residues
   $\varphi(s)-s$ must be $0$.
   Choose such an $s_{0}$.
   Then
          $\varphi(s_{0}) = s_{0}+m$,
   i.e. the procedure that starts at $C_{s_{0}}$
   goes once around the circle, distributes *every one*
   of the $m$ cupcakes, and then stops.

6.  In that distribution each person $P_{k}$ has been given a
   consecutive block of cupcakes whose total score
   (according to $P_{k}$) is at least 1.
   The blocks are pairwise disjoint, so every cupcake has been
   given to exactly one person.

   That is precisely the required assignment of cupcakes.


Why the procedure always reaches the next person  
—————————————————————————————————

Fix a person $P_{k}$ and look at the moment the pointer starts
giving cupcakes to that person.

Because $P_{1},\dots ,P_{k-1}$ already received *consecutive*
blocks, the circle is split into at most $k$ disjoint arcs that
have been removed and one big arc that is still untouched.
The hypothesis of the problem gives $P_{k}$ a personal partition
of the whole circle into $n$ consecutive arcs, each of
score $\ge1$.  At most $k-1$ of those arcs have been touched
already, so at least one complete “untouched’’ arc of the
partition remains.  Its total score is $\ge1$, therefore
$P_{k}$ certainly reaches total score 1 *before* the pointer
travels all the way around the circle.  Hence the procedure
is well‐defined and uses no more than $m$ cupcakes for any one
person.

Therefore every part of the construction is legitimate, and
Steps 1–6 give a rigorous proof that the desired distribution
exists.