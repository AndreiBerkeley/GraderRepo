Label the rows and columns \(1,2,\dots ,n\).
If we choose one square in every row and every column we
necessarily pick the squares  

\[
(1,\sigma(1)),(2,\sigma(2)),\dots ,(n,\sigma(n))
\]

for some permutation \(\sigma\) of \(\{1,\dots ,n\}\).
The number written in the square \((i,\sigma(i))\) is  
\(i+\sigma(i)-k\).

--------------------------------------------------------------------
I.  A necessary condition
--------------------------------------------------------------------
The required numbers are \(1,2,\dots ,n\); their sum is  

\[
1+2+\cdots +n=\frac{n(n+1)}2 .
\]

On the other hand the sum of the numbers in the chosen squares is  

\[
\sum_{i=1}^{n}\bigl(i+\sigma(i)-k\bigr)
      =\sum_{i=1}^{n}i+\sum_{i=1}^{n}\sigma(i)-nk
      =\frac{n(n+1)}2+\frac{n(n+1)}2-nk
      =n(n+1)-nk .
\]

Equality of the two sums gives  

\[
\frac{n(n+1)}2=n(n+1)-nk
\quad\Longrightarrow\quad
n(n+1)-2nk=0
\quad\Longrightarrow\quad
k=\frac{n+1}{2}.
\]

Hence \(k\) must be the half–integer \((n+1)/2\); therefore \(n\)
has to be odd.  Thus

  Necessary condition   \(n\) is odd and \(k=\dfrac{n+1}{2}\).

--------------------------------------------------------------------
II.  A construction when the condition holds
--------------------------------------------------------------------
Now assume \(n\) is odd and put  

\[
k=\frac{n+1}{2}.
\]

Define a permutation \(\sigma\) by a fixed cyclic shift:

\[
\sigma(i)\;=\;
\begin{cases}
i+k,& 1\le i\le n-k,\\[4pt]
i+k-n,& n-k< i\le n.
\end{cases}
\qquad\Bigl(\text{equivalently } \sigma(i)\equiv i+k\;(\bmod n)\Bigr)
\]

Because adding the constant \(k\) modulo \(n\) is a bijection,
\(\sigma\) really is a permutation, so the chosen squares lie in
different rows and columns.

--------------------------------------------------------------------
III.  The values that appear
--------------------------------------------------------------------
Compute \(i+\sigma(i)\).

•  If \(1\le i\le n-k=\dfrac{n-1}{2}\),
   then \(\sigma(i)=i+k\) and  

   \[
   i+\sigma(i)=i+(i+k)=2i+k.
   \]

•  If \(\dfrac{n+1}{2}\le i\le n\),
   write \(i=\dfrac{n+1}{2}+t\;(0\le t\le \tfrac{n-1}{2})\).
   Then \(\sigma(i)=i+k-n\) and  

   \[
   i+\sigma(i)=i+i+k-n
              =2\Bigl(\tfrac{n+1}{2}+t\Bigr)+k-n
              =1+k+2t.
   \]

Thus the set \(\{\,i+\sigma(i)\mid1\le i\le n\,\}\) is

\[
\{\,k+1,k+2,\dots ,k+n\,\}.
\]

Finally, the numbers actually written in the selected squares are

\[
\bigl(i+\sigma(i)-k\bigr)_{i=1}^{n}
  \;=\;\{\,k+1-k,k+2-k,\dots ,k+n-k\,\}
  \;=\;\{\,1,2,\dots ,n\,\}.
\]

--------------------------------------------------------------------
IV.  Conclusion
--------------------------------------------------------------------
An admissible selection exists
exactly when

•  \(n\) is odd, and  
•  \(k=\dfrac{n+1}{2}\).

For these parameters the above permutation furnishes the required
\(n\) squares; for all other pairs \((n,k)\) no such selection can
exist.

Hence the answer is:

Possible ⇔ \(n\) is odd and \(k=(n+1)/2\).