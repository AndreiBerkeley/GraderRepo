% Auto-graded by OpenAI
The student's solution is incorrect.

The first mistake is in the statement "Since the sequence \( 1,a,a^2,\dots,a^{p-5} \) can be rearranged to form a sequence \( b_0,b_1,b_2,\dots,b_{p-5} \), the sequence \( b_0,b_1,b_2,\dots,b_{p-5} \) contains all residues modulo \( p \) except for \( 0 \) (since \( a^0 = 1 \))." This statement is not necessarily true. Just because the sequence \( 1,a,a^2,\dots,a^{p-5} \) can be rearranged to form a sequence \( b_0,b_1,b_2,\dots,b_{p-5} \) does not mean that the sequence \( b_0,b_1,b_2,\dots,b_{p-5} \) contains all residues modulo \( p \) except for \( 0 \).

Therefore, the conclusion that \( p = 2 \) is incorrect. The student's solution does not provide a valid proof for the non-existence of primes \( p \) satisfying the given conditions.

I would rate this solution a 1 out of 5.
