Answer:  No.  
After at most three steps the quadrilaterals begin to repeat, so an
infinite chain of pairwise non–congruent partners cannot exist.

Complete argument  

1.  Encoding the chain by a sequence of points  
   Let  

            \(Q_{0}=P _0P _1P _2P _3 ,\qquad   
            Q_{1}=P _1P _2P _3P _4 ,\qquad   
            Q_{2}=P _2P _3P _4P _5 ,\qquad \dots \)

   be an arbitrary chain of partners, so that for every
   \(k\ge 0\)

            \(P _{k+4}\) is the reflection of \(P _k\) in the
            perpendicular bisector of \(P _{k+1}P _{k+3}\).     (1)

   Convexity of all quadrilaterals implies that the polygon
   \(P _0P _1P _2\dots\) is strictly
   convex and the points occur in this order around the polygon.

2.  Three–periodicity of the successive edges  
   Put  

            \(d_k=\lvert P _kP _{k+1}\rvert\;\;(k\ge 0).\)

   The reflection that appears in (1) interchanges  
   \(P _{k+1}\) and \(P _{k+3}\); hence it carries the segment
   \(P _kP _{k+1}\) on to the segment \(P _{k+4}P _{k+3}\).
   Being an isometry, it preserves lengths, and therefore  

            \(d_k=d_{k+3}\qquad(k\ge 0).\)                                (2)

   Thus the infinite sequence \(d_0,d_1,d_2,d_3,\dots\)
   assumes at most three different values:
   \(d_0,d_1,d_2\).

3.  Constancy of the fourth side  
   Denote  

            \(s_k=\lvert P _{k+3}P _k\rvert \)  (the side that closes
   \(Q_k\)).  

   Equation (1) shows that the reflection in the perpendicular
   bisector of \(P _{k+1}P _{k+3}\) sends \(P _k\) to \(P _{k+4}\)
   and \(P _{k+3}\) to \(P _{k+1}\); hence  

            \(s_k=\lvert P _{k+3}P _k\rvert
                =\lvert P _{k+1}P _{k+4}\rvert=s_{k+1}.\)

   So  

            \(s_k\equiv s\) is the same constant for every \(k\).     (3)

4.  The quadrilateral \(Q_{k+3}\) has the same five metric data as
    \(Q_k\)  

   The four sides of \(Q_k=P _kP _{k+1}P _{k+2}P _{k+3}\) are  

            \(d_k,\;d_{k+1},\;d_{k+2},\;s,\)

   while the four sides of \(Q_{k+3}=P _{k+3}P _{k+4}P _{k+5}P _{k+6}\)
   are, by (2) and (3),

            \(d_{k+3}=d_k,\;d_{k+4}=d_{k+1},\;
             d_{k+5}=d_{k+2},\;s_{k+3}=s.\)

   Moreover the two quadrilaterals share the diagonal  
   \(P _{k+1}P _{k+3}\); hence they also have a common diagonal length.
   Consequently \(Q_{k}\) and \(Q_{k+3}\) possess the same
   set of five lengths

            \(d_k,\;d_{k+1},\;d_{k+2},\;s,\;\lvert P _{k+1}P _{k+3}\rvert .\)

5.  These data determine a convex quadrilateral uniquely  
   Given the five lengths just listed one constructs the triangle
   \(P _kP _{k+1}P _{k+3}\) from the three sides
   \(d_k,s,\lvert P _{k+1}P _{k+3}\rvert\);   
   there is only one way (up to congruence) to attach the point
   \(P _{k+2}\) because
   the sides with lengths \(d_{k+1}\) and \(d_{k+2}\) must lie on
   opposite sides of the diagonal \(P _{k+1}P _{k+3}\) in order to keep
   the quadrilateral convex.
   Hence the five lengths fix the quadrilateral up to a rigid motion.

   Thus  

            \(Q_{k+3}\cong Q_k\qquad(k\ge 0).\)                        (4)

6.  End of the proof  
   Relation (4) shows that only the three quadrilaterals
   \(Q_0, Q_1, Q_2\) are possible up to congruence.
   Therefore an infinite sequence in which every member is a partner
   of the next must sooner or later repeat a congruence class,
   which is forbidden by the statement of the problem.  
   Consequently no such infinite sequence exists.