Let $V$ be the set of all linear combinations of $v_1, \dots, v_{12}$ with integer coefficients. We want to show that $V$ is dense in $\mathbb{R}^3$. 

Since the vectors $v_1, \dots, v_{12}$ are unit vectors, they lie on the unit sphere centered at the origin. The regular icosahedron is a convex polyhedron inscribed in this unit sphere. Since the icosahedron is a convex set, the convex hull of its vertices is the entire icosahedron itself. Thus, any point on the unit sphere can be written as a convex combination of the vertices of the icosahedron.

Let $v \in \mathbb{R}^3$ be an arbitrary vector. Since $v$ lies on the unit sphere, it can be written as a convex combination of the vertices of the icosahedron:
\[
v = \lambda_1 v_1 + \lambda_2 v_2 + \cdots + \lambda_{12} v_{12},
\]
where $\lambda_1, \lambda_2, \dots, \lambda_{12}$ are non-negative real numbers that sum to $1$. 

Now, consider the vector
\[
w = \lambda_1 v_1 + \lambda_2 v_2 + \cdots + \lambda_{12} v_{12} - v.
\]
Since $v$ is a convex combination of the vertices of the icosahedron, $w$ is a linear combination of the vectors $v_1, \dots, v_{12}$ with non-negative coefficients that sum to $-1$. In other words, $w$ lies in the cone generated by the vectors $v_1, \dots, v_{12}$.

For any $\varepsilon > 0$, we can find a vector $w' \in V$ such that $\|w - w'\| < \varepsilon$. This is because $V$ is a lattice in $\mathbb{R}^3$ (since it consists of all integer linear combinations of $v_1, \dots, v_{12}$), and hence is dense in $\mathbb{R}^3$. 

Therefore, we can write $w'$ as
\[
w' = a_1 v_1 + a_2 v_2 + \cdots + a_{12} v_{12},
\]
where $a_1, a_2, \dots, a_{12}$ are integers. 

Putting it all together, we have
\[
\| a_1 v_1 + a_2 v_2 + \cdots + a_{12} v_{12} - v \| = \| w' - w \| < \varepsilon,
\]
which completes the proof.
