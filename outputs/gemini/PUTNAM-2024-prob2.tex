Of course. Here is a complete and rigorous step-by-step solution to the problem.

### Problem
Find the number of ordered pairs of integers $(x, y)$ that satisfy the equation $x^3 + y^3 - 3xy = -1$.

### Solution

The given equation is a cubic Diophantine equation:
$$x^3 + y^3 - 3xy = -1$$

**Step 1: Rearrange the equation and apply a key algebraic identity.**

We can rewrite the equation by moving the constant term to the left side:
$$x^3 + y^3 + 1 - 3xy = 0$$

This expression is reminiscent of the well-known factorization identity:
$$a^3 + b^3 + c^3 - 3abc = (a+b+c)(a^2+b^2+c^2-ab-bc-ca)$$

To apply this identity to our equation, we can set $a=x$, $b=y$, and $c=1$. The equation becomes:
$$x^3 + y^3 + 1^3 - 3x \cdot y \cdot 1 = 0$$

Using the identity, we can factor the left side:
$$(x+y+1)(x^2+y^2+1^2 - xy - y(1) - (1)x) = 0$$
$$(x+y+1)(x^2+y^2-xy-x-y+1) = 0$$

For this product of two factors to be zero, at least one of the factors must be zero. Since $x$ and $y$ are integers, both factors $(x+y+1)$ and $(x^2+y^2-xy-x-y+1)$ must also be integers. This leads to two separate cases.

**Case 1: The first factor is zero.**
$$x+y+1 = 0$$
This is a linear equation. We can express $y$ in terms of $x$:
$$y = -x - 1$$
For any integer value of $x$, the corresponding value of $y$ will also be an integer. Let's choose an arbitrary integer $k$ for $x$.
If $x = k$, where $k \in \mathbb{Z}$, then $y = -k - 1$.
Thus, all ordered pairs of the form $(k, -k-1)$ for any integer $k$ are solutions to the equation.

For example:
- If $k=0$, $(x,y) = (0, -1)$. Check: $0^3 + (-1)^3 - 3(0)(-1) = -1$.
- If $k=1$, $(x,y) = (1, -2)$. Check: $1^3 + (-2)^3 - 3(1)(-2) = 1 - 8 + 6 = -1$.
- If $k=-1$, $(x,y) = (-1, 0)$. Check: $(-1)^3 + 0^3 - 3(-1)(0) = -1$.

Since there are infinitely many integers $k$, this case yields an infinite number of solutions.

**Case 2: The second factor is zero.**
$$x^2 + y^2 - xy - x - y + 1 = 0$$
This is a quadratic equation in two variables. To find the integer solutions, we can rearrange this equation into a sum of squares. Multiply the entire equation by 2 to facilitate this:
$$2x^2 + 2y^2 - 2xy - 2x - 2y + 2 = 0$$
Now, we can group the terms to form perfect squares:
$$(x^2 - 2xy + y^2) + (x^2 - 2x + 1) + (y^2 - 2y + 1) = 0$$
Factoring each group gives:
$$(x-y)^2 + (x-1)^2 + (y-1)^2 = 0$$
We are looking for integer solutions $(x, y)$. For any integers $x$ and $y$, the terms $(x-y)$, $(x-1)$, and $(y-1)$ are also integers. The square of any integer is non-negative. The sum of non-negative numbers can only be zero if each number is zero.
Therefore, we must have:
1.  $(x-y)^2 = 0 \implies x-y = 0 \implies x=y$
2.  $(x-1)^2 = 0 \implies x-1 = 0 \implies x=1$
3.  $(y-1)^2 = 0 \implies y-1 = 0 \implies y=1$

From (2) and (3), we get $x=1$ and $y=1$. This is consistent with condition (1), since if $x=1$ and $y=1$, then $x=y$.
Thus, the only integer solution from this case is $(1, 1)$.
We can check this solution in the original equation: $1^3 + 1^3 - 3(1)(1) = 1 + 1 - 3 = -1$. This is correct.

**Step 3: Combine the results from both cases.**

The set of all integer solutions is the union of the solutions from Case 1 and Case 2.
- The solutions from Case 1 are of the form $(k, -k-1)$ for any integer $k$.
- The solution from Case 2 is $(1, 1)$.

We should check if the solution from Case 2 is already included in Case 1. For $(1, 1)$ to be a solution from Case 1, it must satisfy $x+y+1=0$.
$1+1+1 = 3 \neq 0$.
So, the solution $(1, 1)$ is not part of the family of solutions from Case 1. The two sets of solutions are disjoint.

The total set of solutions is $\{(k, -k-1) \mid k \in \mathbb{Z}\} \cup \{(1, 1)\}$.
Since the first set is infinite, the total number of ordered pairs of integers $(x,y)$ that satisfy the given equation is infinite.

### Final Answer
The equation factors into $(x+y+1)(x^2+y^2-xy-x-y+1) = 0$.
This implies either $x+y+1=0$ or $x^2+y^2-xy-x-y+1=0$.
The first case, $x+y+1=0$, gives an infinite family of integer solutions $(k, -k-1)$ for any integer $k$.
The second case, $x^2+y^2-xy-x-y+1=0$, can be rewritten as $(x-y)^2+(x-1)^2+(y-1)^2=0$, which has the unique integer solution $(1,1)$.
Since there is an infinite family of solutions from the first case, the total number of solutions is **infinite**.