This is a two-part problem. First, we must show that the polynomial $P(x) = x^3 - 3x^2 + 1$ has three distinct real roots. Second, we must calculate the value of a given expression involving these roots and the roots of a related polynomial.

### Part 1: Existence of three distinct real roots

Let the given polynomial be $P(x) = x^3 - 3x^2 + 1$. To determine the number of real roots, we can analyze the behavior of the function $P(x)$ using calculus.

1.  **Find the critical points:** We start by finding the derivative of $P(x)$ and setting it to zero to find the critical points.
    $P'(x) = \frac{d}{dx}(x^3 - 3x^2 + 1) = 3x^2 - 6x$.
    Setting $P'(x) = 0$:
    $3x^2 - 6x = 0$
    $3x(x - 2) = 0$
    The critical points are $x=0$ and $x=2$.

2.  **Determine the nature of the critical points:** We can use the second derivative test to classify these points as local maxima or minima.
    $P''(x) = \frac{d}{dx}(3x^2 - 6x) = 6x - 6$.
    *   At $x=0$: $P''(0) = 6(0) - 6 = -6 < 0$, which indicates a local maximum.
    *   At $x=2$: $P''(2) = 6(2) - 6 = 6 > 0$, which indicates a local minimum.

3.  **Evaluate the function at the critical points:**
    *   The local maximum value is $P(0) = (0)^3 - 3(0)^2 + 1 = 1$.
    *   The local minimum value is $P(2) = (2)^3 - 3(2)^2 + 1 = 8 - 12 + 1 = -3$.

4.  **Analyze the end behavior and apply the Intermediate Value Theorem:**
    *   As $x \to -\infty$, $P(x) \to -\infty$.
    *   As $x \to \infty$, $P(x) \to \infty$.

    Now we combine these results:
    *   The function comes from $-\infty$, increases to a local maximum of $P(0)=1$. Since it goes from negative values to a positive value, there must be a root $x_1$ in the interval $(-\infty, 0)$.
    *   The function then decreases from the local maximum $P(0)=1$ to the local minimum $P(2)=-3$. Since it goes from a positive value to a negative value, there must be a second root $x_2$ in the interval $(0, 2)$.
    *   Finally, the function increases from the local minimum $P(2)=-3$ towards $\infty$. Since it starts from a negative value and increases indefinitely, there must be a third root $x_3$ in the interval $(2, \infty)$.

Since we have found three roots in three disjoint intervals, the three real roots $x_1, x_2, x_3$ are distinct.

### Part 2: Evaluating the expression

Let $x_1, x_2, x_3$ be the three distinct real roots of $P(x) = x^3 - 3x^2 + 1 = 0$.
Let $y_1, y_2, y_3$ be the roots of the equation $P(y-1) = 0$.

1.  **Relating the roots $y_i$ to $x_i$:**
    The equation $P(y-1)=0$ means that $y-1$ is a value that, when plugged into $P$, gives zero. By definition, the values for which $P$ is zero are the roots $x_1, x_2, x_3$.
    Therefore, the set of values $\{y_1-1, y_2-1, y_3-1\}$ must be the same as the set of roots $\{x_1, x_2, x_3\}$.
    This implies that the roots are related by $y_i - 1 = x_j$ for some pairing of indices $(i, j)$.
    So, the set of roots of $P(y-1)=0$ is $\{y_1, y_2, y_3\} = \{x_1+1, x_2+1, x_3+1\}$.

2.  **Pairing of the roots:**
    The problem asks for the value of $(x_1 + y_1)(x_2 + y_2)(x_3 + y_3)$. This expression's value could depend on how we pair the $x_i$'s with the $y_i$'s. A standard interpretation in such problems is that the indices correspond. We can establish this correspondence formally. Let's order the roots of $P(x)$ as $x_1 < x_2 < x_3$.
    Let $f(t) = t+1$. The roots of $P(y-1)=0$ are $y_i = f(x_j)$ for some permutation. Since $f(t)$ is a strictly increasing function, we have $x_1+1 < x_2+1 < x_3+1$. If we order the roots $y_i$ as $y_1 < y_2 < y_3$, the natural and unique pairing is $y_i = x_i+1$ for $i=1, 2, 3$.
    We assume this natural pairing, so $y_1 = x_1+1$, $y_2 = x_2+1$, and $y_3 = x_3+1$.

3.  **Calculating the expression:**
    We need to find the value of $E = (x_1 + y_1)(x_2 + y_2)(x_3 + y_3)$.
    Substituting $y_i = x_i+1$:
    $E = (x_1 + (x_1+1))(x_2 + (x_2+1))(x_3 + (x_3+1))$
    $E = (2x_1 + 1)(2x_2 + 1)(2x_3 + 1)$

    To evaluate this expression, we can relate it to the polynomial $P(x)$.
    Since $x_1, x_2, x_3$ are the roots of $P(x)$, we can write $P(x)$ in factored form:
    $P(x) = (x-x_1)(x-x_2)(x-x_3)$.

    Let's manipulate the expression for $E$:
    $E = (2x_1 + 1)(2x_2 + 1)(2x_3 + 1)$
    $E = 2\left(x_1 + \frac{1}{2}\right) \cdot 2\left(x_2 + \frac{1}{2}\right) \cdot 2\left(x_3 + \frac{1}{2}\right)$
    $E = 8\left(x_1 - \left(-\frac{1}{2}\right)\right)\left(x_2 - \left(-\frac{1}{2}\right)\right)\left(x_3 - \left(-\frac{1}{2}\right)\right)$

    This expression is almost in the form of $P(x)$ evaluated at a specific point. Let's adjust the signs:
    $E = 8 \cdot (-1)^3 \left(\left(-\frac{1}{2}\right) - x_1\right)\left(\left(-\frac{1}{2}\right) - x_2\right)\left(\left(-\frac{1}{2}\right) - x_3\right)$
    $E = -8 \left[\left(\left(-\frac{1}{2}\right) - x_1\right)\left(\left(-\frac{1}{2}\right) - x_2\right)\left(\left(-\frac{1}{2}\right) - x_3\right)\right]$

    The term in the square brackets is exactly the definition of $P(-1/2)$.
    So, $E = -8 \cdot P(-1/2)$.

    Now, we just need to calculate $P(-1/2)$:
    $P\left(-\frac{1}{2}\right) = \left(-\frac{1}{2}\right)^3 - 3\left(-\frac{1}{2}\right)^2 + 1$
    $P\left(-\frac{1}{2}\right) = -\frac{1}{8} - 3\left(\frac{1}{4}\right) + 1$
    $P\left(-\frac{1}{2}\right) = -\frac{1}{8} - \frac{3}{4} + 1$
    $P\left(-\frac{1}{2}\right) = -\frac{1}{8} - \frac{6}{8} + \frac{8}{8}$
    $P\left(-\frac{1}{2}\right) = \frac{1}{8}$

    Finally, we can compute $E$:
    $E = -8 \cdot P(-1/2) = -8 \cdot \left(\frac{1}{8}\right) = -1$.

### Alternative Method using Vieta's Formulas

We can also calculate $E = (2x_1 + 1)(2x_2 + 1)(2x_3 + 1)$ by expanding it and using Vieta's formulas.
For $P(x) = x^3 - 3x^2 + 0x + 1 = 0$, Vieta's formulas give us the sums of the products of the roots:
*   $x_1 + x_2 + x_3 = -(-3)/1 = 3$
*   $x_1x_2 + x_1x_3 + x_2x_3 = 0/1 = 0$
*   $x_1x_2x_3 = -1/1 = -1$

Expanding the expression for $E$:
$E = (2x_1+1)(2x_2+1)(2x_3+1)$
$E = (4x_1x_2 + 2x_1 + 2x_2 + 1)(2x_3 + 1)$
$E = 8x_1x_2x_3 + 4x_1x_2(1) + 4x_1x_3(1) + 4x_2x_3(1) + 2x_1(1) + 2x_2(1) + 2x_3(1) + 1$
$E = 8(x_1x_2x_3) + 4(x_1x_2 + x_1x_3 + x_2x_3) + 2(x_1 + x_2 + x_3) + 1$

Now, substitute the values from Vieta's formulas:
$E = 8(-1) + 4(0) + 2(3) + 1$
$E = -8 + 0 + 6 + 1$
$E = -1$

Both methods yield the same result.

### Conclusion

We have shown that $P(x)=x^3-3x^2+1$ has three distinct real roots. By establishing a natural pairing between the roots of $P(x)=0$ and $P(y-1)=0$, we found the value of the expression $(x_1 + y_1)(x_2 + y_2)(x_3 + y_3)$ to be **-1**.